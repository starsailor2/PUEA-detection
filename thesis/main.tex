\documentclass[12pt,a4paper]{report}
\usepackage{amsmath,amssymb,amsfonts}
\usepackage{graphicx}
\usepackage{booktabs}
\usepackage{hyperref}
\usepackage{algorithm}
\usepackage{algpseudocode}
\usepackage{subcaption}
\usepackage{url}
\usepackage{listings}
\usepackage[T1]{fontenc}
\usepackage{setspace}
\usepackage{fancyhdr}

\hypersetup{
    colorlinks=true,
    linkcolor=blue,
    filecolor=magenta,      
    urlcolor=cyan,
    pdftitle={Primary User Emulation Attack Detection in Cognitive Radio Networks},
    pdfpagemode=FullScreen,
}

\title{Primary User Emulation Attack Detection in Cognitive Radio Networks\\
\large Using Machine Learning Clustering Techniques}
\author{Your Name}
\date{\today}

\begin{document}

\maketitle
\tableofcontents
\listoffigures
\listoftables

\chapter{Introduction}
\section{Background}
Cognitive Radio Networks (CRN) have emerged as a promising solution to address the spectrum scarcity issue by enabling dynamic spectrum access. In a CRN, secondary users (SUs) are allowed to opportunistically access the spectrum bands when the primary users (PUs) are not using them. However, this opportunistic access paradigm introduces new security challenges, one of which is the Primary User Emulation Attack (PUEA).

A Primary User Emulation Attack occurs when a malicious entity mimics the signal characteristics of a legitimate primary user to deceive secondary users into vacating the spectrum. This attack can significantly degrade the performance of cognitive radio networks by denying service to legitimate SUs and disrupting the spectrum sensing process. Therefore, developing effective methods to detect and mitigate PUEA is crucial for the secure operation of cognitive radio networks.

\section{Problem Statement}
The accurate detection of Primary User Emulation Attackers presents several challenges due to the similarity between legitimate PU signals and PUEA signals. Traditional detection methods that rely solely on energy detection or signal features may not be sufficient to distinguish between legitimate PUs and PUEAs. Moreover, the dynamic nature of wireless environments, including shadowing, fading, and path loss effects, further complicates the detection process.

This thesis addresses the challenge of effectively detecting PUEAs in cognitive radio networks by developing a machine learning-based approach that utilizes clustering algorithms to identify and separate PUEA signals from legitimate PU signals based on their transmission characteristics.

\section{Research Objectives}
The main objectives of this research are:
\begin{enumerate}
    \item To generate a realistic dataset of PU and PUEA signals considering various signal propagation factors.
    \item To develop a detection model using clustering algorithms (DBSCAN, Agglomerative, K-means) to effectively identify PUEAs.
    \item To evaluate and compare the performance of different clustering algorithms for PUEA detection.
    \item To propose an iterative clustering method that improves detection accuracy.
\end{enumerate}

\section{Thesis Organization}
The remainder of this thesis is organized as follows:
\begin{itemize}
    \item Chapter 2 presents a comprehensive literature review on cognitive radio networks, security threats, primary user emulation attacks, and existing detection techniques.
    \item Chapter 3 details the methodology, including dataset generation, feature selection, and the proposed clustering-based detection approach.
    \item Chapter 4 provides implementation details and experimental setup.
    \item Chapter 5 presents the results and performance analysis.
    \item Chapter 6 concludes the thesis with a summary of findings and suggestions for future work.
\end{itemize}

% Include the chapter files
\chapter{Literature Review}

\section{Cognitive Radio Networks}
Cognitive Radio Networks (CRNs) represent an evolution in wireless communication systems that aims to address the growing problem of spectrum scarcity. Introduced by Mitola \cite{mitola1999cognitive} in 1999, cognitive radio technology enables intelligent and dynamic spectrum usage by allowing secondary users to access licensed spectrum bands when they are not being utilized by primary users. This opportunistic spectrum access paradigm is facilitated through key capabilities such as spectrum sensing, spectrum decision, spectrum sharing, and spectrum mobility \cite{akyildiz2006next}.

\subsection{Spectrum Sensing}
Spectrum sensing is the fundamental capability that allows cognitive radios to detect spectrum holes or white spaces. Various techniques have been developed for spectrum sensing, including energy detection, matched filter detection, cyclostationary feature detection, and cooperative sensing \cite{yucek2009survey}. Each of these techniques offers different trade-offs between detection accuracy, computational complexity, and implementation requirements.

\subsection{Dynamic Spectrum Access}
Dynamic Spectrum Access (DSA) enables secondary users to opportunistically access spectrum bands without causing harmful interference to primary users. DSA systems implement policies and protocols that govern when and how secondary users can access the spectrum, ensuring that primary users' communication rights are protected \cite{zhao2007survey}.

\section{Security Threats in Cognitive Radio Networks}
Despite their numerous benefits, CRNs are vulnerable to various security threats due to their open and dynamic nature. These threats can be classified into different categories based on the targeted layer of the network architecture \cite{fragkiadakis2013survey}.

\subsection{Physical Layer Threats}
Physical layer threats in CRNs include jamming attacks, primary user emulation attacks, and objective function attacks. These attacks directly target the fundamental operations of cognitive radios, particularly spectrum sensing and decision-making processes \cite{wang2010security}.

\subsection{MAC Layer Threats}
Medium Access Control (MAC) layer threats include control channel saturation, biasing sensing results in cooperative sensing, and selfish behavior in spectrum sharing. These attacks can disrupt the coordination among cognitive radio users and lead to inefficient spectrum utilization \cite{leon2010security}.

\section{Primary User Emulation Attacks}
Primary User Emulation Attack (PUEA) is one of the most severe threats to cognitive radio networks where a malicious entity mimics the signal characteristics of a legitimate primary user \cite{chen2008defense}.

\subsection{PUEA Models}
PUEA can be categorized into two main models: selfish PUEA and malicious PUEA \cite{chen2008defense}. In selfish PUEA, the attacker's objective is to maximize its own spectrum usage by forcing other secondary users to vacate the spectrum. In contrast, malicious PUEA aims to disrupt the entire cognitive radio network operation without necessarily benefiting from the spectrum usage.

\subsection{Impact of PUEA}
The impact of PUEA on cognitive radio networks is multifaceted. First, it reduces spectrum utilization efficiency by denying secondary users access to available spectrum bands. Second, it increases the false alarm rate in spectrum sensing, leading to unnecessary spectrum handoffs and service disruptions. Third, it can cause denial of service for legitimate secondary users, resulting in degraded network performance \cite{jin2010advanced}.

\section{PUEA Detection Techniques}
Various approaches have been proposed in the literature to detect and mitigate PUEAs. These approaches can be broadly classified into location-based, signal feature-based, and machine learning-based techniques.

\subsection{Location-based Detection}
Location-based detection techniques exploit the spatial information of transmitters to distinguish between legitimate primary users and attackers \cite{chen2008defense}. These techniques often rely on the deployment of location verification infrastructure, such as anchor nodes or direction-finding systems, to verify the claimed location of signal sources.

\subsection{Signal Feature-based Detection}
Signal feature-based detection techniques analyze various characteristics of received signals, such as energy level, modulation type, cyclic features, and channel impulse response, to identify anomalies that may indicate the presence of PUEAs \cite{yu2011optimal}.

\subsection{Machine Learning-based Detection}
Machine learning techniques have gained significant attention for PUEA detection due to their ability to learn from data and adapt to changing environments. Various supervised, unsupervised, and semi-supervised learning algorithms have been applied to this problem \cite{kawalec2019application}.

\subsubsection{Supervised Learning Approaches}
Supervised learning approaches for PUEA detection include support vector machines (SVM), artificial neural networks (ANN), and decision trees. These approaches require labeled training data containing both legitimate PU signals and PUEA signals \cite{tang2015spectrumwatch}.

\subsubsection{Unsupervised Learning Approaches}
Unsupervised learning approaches, such as clustering algorithms, do not require labeled training data and can identify natural groupings or patterns in signal data. K-means, DBSCAN, and hierarchical clustering have been applied to detect anomalies in spectrum sensing data that may indicate the presence of PUEAs \cite{rajasekar2015detection}.

\subsubsection{Semi-supervised Learning Approaches}
Semi-supervised learning approaches combine elements of both supervised and unsupervised learning, typically using a small amount of labeled data along with a larger amount of unlabeled data. These approaches are particularly suitable for PUEA detection in real-world scenarios where obtaining labeled data is challenging \cite{abdelaziz2018semi}.

\section{Clustering Algorithms for Anomaly Detection}
Clustering algorithms have been widely used for anomaly detection in various domains, including network security. In the context of PUEA detection, clustering algorithms can help identify outliers or unusual signal patterns that deviate from the normal behavior of legitimate primary users.

\subsection{K-means Clustering}
K-means clustering partitions observations into k clusters in which each observation belongs to the cluster with the nearest mean \cite{hartigan1979algorithm}. In PUEA detection, K-means can be used to separate legitimate PU signals from PUEA signals based on their feature vectors.

\subsection{DBSCAN Algorithm}
Density-Based Spatial Clustering of Applications with Noise (DBSCAN) is a density-based clustering algorithm that groups together points that are closely packed while marking points in low-density regions as outliers \cite{ester1996density}. DBSCAN is particularly suitable for PUEA detection as it can identify anomalies without requiring the number of clusters to be specified in advance.

\subsection{Agglomerative Hierarchical Clustering}
Agglomerative hierarchical clustering builds a hierarchy of clusters by progressively merging clusters \cite{murtagh2012algorithms}. This approach allows for the analysis of the cluster structure at different levels of granularity, which can be beneficial for understanding the relationship between different types of signals in PUEA detection.

\section{Research Gap and Contribution}
While various approaches have been proposed for PUEA detection, there is still a need for more effective and reliable detection methods, particularly in dynamic and uncertain wireless environments. Most existing works focus on specific aspects of PUEA detection and often make simplified assumptions about signal propagation models or attack scenarios.

This research contributes to the field by developing a comprehensive PUEA detection framework that:
\begin{enumerate}
    \item Generates realistic datasets incorporating multiple signal propagation factors.
    \item Applies and compares different clustering algorithms for PUEA detection.
    \item Proposes an iterative clustering method to improve detection accuracy.
    \item Evaluates the proposed approach under various attack scenarios and environmental conditions.
\end{enumerate}

% Add references section
\begin{thebibliography}{99}
\bibitem{mitola1999cognitive} Mitola, J., and Maguire, G. Q. (1999). Cognitive radio: making software radios more personal. IEEE Personal Communications, 6(4), 13-18.

\bibitem{akyildiz2006next} Akyildiz, I. F., Lee, W. Y., Vuran, M. C., and Mohanty, S. (2006). Next generation/dynamic spectrum access/cognitive radio wireless networks: A survey. Computer Networks, 50(13), 2127-2159.

\bibitem{yucek2009survey} Yucek, T., and Arslan, H. (2009). A survey of spectrum sensing algorithms for cognitive radio applications. IEEE Communications Surveys \& Tutorials, 11(1), 116-130.

\bibitem{zhao2007survey} Zhao, Q., and Sadler, B. M. (2007). A survey of dynamic spectrum access. IEEE Signal Processing Magazine, 24(3), 79-89.

\bibitem{fragkiadakis2013survey} Fragkiadakis, A. G., Tragos, E. Z., and Askoxylakis, I. G. (2013). A survey on security threats and detection techniques in cognitive radio networks. IEEE Communications Surveys \& Tutorials, 15(1), 428-445.

\bibitem{wang2010security} Wang, W., Li, H., Sun, Y., and Han, Z. (2010). Security in cognitive radio networks: An overview. In 2010 IEEE Wireless Communication and Networking Conference (pp. 1-6). IEEE.

\bibitem{leon2010security} Leon, O., Hernandez-Serrano, J., and Soriano, M. (2010). Securing cognitive radio networks. International Journal of Communication Systems, 23(5), 633-652.

\bibitem{chen2008defense} Chen, R., Park, J. M., and Reed, J. H. (2008). Defense against primary user emulation attacks in cognitive radio networks. IEEE Journal on Selected Areas in Communications, 26(1), 25-37.

\bibitem{jin2010advanced} Jin, Z., Anand, S., and Subbalakshmi, K. P. (2010). Detecting primary user emulation attacks in dynamic spectrum access networks. In 2010 IEEE International Conference on Communications (pp. 1-5). IEEE.

\bibitem{yu2011optimal} Yu, F. R., Tang, H., Huang, M., Li, Z., and Mason, P. C. (2011). Defense against spectrum sensing data falsification attacks in mobile ad hoc networks with cognitive radios. In MILCOM 2011 Military Communications Conference (pp. 1241-1246). IEEE.

\bibitem{kawalec2019application} Kawalec, A., and Owczarek, R. (2019). Application of machine learning in cognitive radio networks. In 2019 Signal Processing: Algorithms, Architectures, Arrangements, and Applications (SPA) (pp. 127-132). IEEE.

\bibitem{tang2015spectrumwatch} Tang, L., Chen, Y., Hines, E. L., and Alouini, M. S. (2015). Spectrumwatch: A framework for primary user emulation attack detection in cognitive radio networks. In 2015 IEEE International Conference on Communication Workshop (ICCW) (pp. 1573-1578). IEEE.

\bibitem{rajasekar2015detection} Rajasekar, S., Karuppiah, M., and Manoharan, R. (2015). Detection of primary user emulation attacks in cognitive radio networks using support vector machine. In 2015 International Conference on Computing and Communications Technologies (ICCCT) (pp. 165-169). IEEE.

\bibitem{abdelaziz2018semi} Abdelaziz, A., Elsabrouty, M., and Muta, O. (2018). Semi-supervised learning for primary user emulation attack detection using spectral correlation. In 2018 IEEE Global Conference on Signal and Information Processing (GlobalSIP) (pp. 1-5). IEEE.

\bibitem{hartigan1979algorithm} Hartigan, J. A., and Wong, M. A. (1979). Algorithm AS 136: A k-means clustering algorithm. Journal of the Royal Statistical Society. Series C (Applied Statistics), 28(1), 100-108.

\bibitem{ester1996density} Ester, M., Kriegel, H. P., Sander, J., and Xu, X. (1996). A density-based algorithm for discovering clusters in large spatial databases with noise. In Kdd (Vol. 96, No. 34, pp. 226-231).

\bibitem{murtagh2012algorithms} Murtagh, F., and Contreras, P. (2012). Algorithms for hierarchical clustering: an overview. Wiley Interdisciplinary Reviews: Data Mining and Knowledge Discovery, 2(1), 86-97.
\end{thebibliography}

\chapter{Methodology}

\section{Overview of the Proposed Approach}
This chapter presents the methodology for detecting Primary User Emulation Attacks (PUEA) in Cognitive Radio Networks using clustering-based approaches. The proposed methodology consists of four main stages: (1) dataset generation, (2) feature extraction and preprocessing, (3) clustering-based detection, and (4) performance evaluation. Figure \ref{fig:methodology} illustrates the overall architecture of the proposed approach.

% You would need to add a figure here showing the methodology framework

\section{Dataset Generation}
Since real-world datasets for PUEA detection are limited, we generate a synthetic dataset that reflects the signal propagation characteristics of both legitimate primary users (PUs) and PUEAs. The dataset generation process considers various signal propagation factors that affect the received signal power at secondary users (SUs).

\subsection{Signal Propagation Model}
The received power at a secondary user from a transmitter (either a legitimate PU or a PUEA) can be modeled using the log-normal shadowing path loss model \cite{rappaport1996wireless}:

\begin{equation}
P_r = P_t - 10 \alpha \log_{10}\left(\frac{d}{d_0}\right) + X_\sigma
\end{equation}

where:
\begin{itemize}
    \item $P_r$ is the received power at the secondary user (in dBm)
    \item $P_t$ is the transmit power (in dBm)
    \item $\alpha$ is the path loss exponent
    \item $d$ is the distance between the transmitter and receiver (in meters)
    \item $d_0$ is the reference distance (typically 1 meter)
    \item $X_\sigma$ is the shadowing effect, modeled as a zero-mean Gaussian random variable with standard deviation $\sigma$ (in dB)
\end{itemize}

\subsection{Dataset Parameters}
Our dataset consists of five key components for each data point:
\begin{enumerate}
    \item Transmitted power ($P_t$): The power at which the signal is transmitted by either a PU or PUEA.
    \item Distance ($d$): The distance between the transmitter (PU or PUEA) and the receiving secondary user.
    \item Path loss exponent ($\alpha$): Characterizes the environment in which the signal propagates.
    \item Shadowing effect ($X_\sigma$): Represents the random fluctuations in received signal power due to obstacles.
    \item Received power ($P_r$): The power of the signal as measured by the secondary user.
\end{enumerate}

\subsection{Generating PU and PUEA Signals}
To create a realistic dataset, we consider the following approach:

\begin{enumerate}
    \item \textbf{Legitimate PU signals}: 
    \begin{itemize}
        \item Transmit power: Fixed at a standard level (e.g., 30 dBm)
        \item Location: Placed at known, fixed locations
        \item Path loss exponent: Based on the environment (e.g., 2 for free space, 4-6 for urban environments)
    \end{itemize}
    
    \item \textbf{PUEA signals}: 
    \begin{itemize}
        \item Transmit power: Variable, potentially higher than PUs to maximize attack impact
        \item Location: Random locations, different from PU locations
        \item Path loss exponent: Similar to PUs but with potential variations
    \end{itemize}
\end{enumerate}

The following Python code generates the synthetic dataset:

\begin{lstlisting}[language=Python, caption=Dataset Generation Code]
import numpy as np
import pandas as pd
import matplotlib.pyplot as plt
from sklearn.preprocessing import StandardScaler

# Set random seed for reproducibility
np.random.seed(42)

# Function to calculate received power using the log-normal shadowing model
def calculate_received_power(transmit_power, distance, path_loss_exponent, shadowing_std):
    # Reference distance (in meters)
    d0 = 1.0
    
    # Calculate path loss
    path_loss = 10 * path_loss_exponent * np.log10(distance / d0)
    
    # Generate random shadowing effect
    shadowing = np.random.normal(0, shadowing_std)
    
    # Calculate received power
    received_power = transmit_power - path_loss + shadowing
    
    return received_power

# Parameters for generating the dataset
num_pu_samples = 300  # Number of legitimate PU signal samples
num_puea_samples = 150  # Number of PUEA signal samples

# Parameters for legitimate PU signals
pu_transmit_power_mean = 30  # dBm
pu_transmit_power_std = 1  # Small variation in transmit power
pu_distance_mean = 500  # meters
pu_distance_std = 100
pu_path_loss_exponent_mean = 3.0
pu_path_loss_exponent_std = 0.2
pu_shadowing_std = 4  # dB

# Parameters for PUEA signals
puea_transmit_power_mean = 35  # dBm (higher than PU)
puea_transmit_power_std = 2
puea_distance_mean = 450  # meters (potentially closer to SUs)
puea_distance_std = 120
puea_path_loss_exponent_mean = 2.8  # Potentially different environment
puea_path_loss_exponent_std = 0.3
puea_shadowing_std = 5  # dB (potentially more variable)

# Generate legitimate PU signal data
pu_transmit_power = np.random.normal(pu_transmit_power_mean, pu_transmit_power_std, num_pu_samples)
pu_distance = np.random.normal(pu_distance_mean, pu_distance_std, num_pu_samples)
pu_path_loss_exponent = np.random.normal(pu_path_loss_exponent_mean, pu_path_loss_exponent_std, num_pu_samples)
pu_shadowing = np.random.normal(0, pu_shadowing_std, num_pu_samples)

# Calculate received power for legitimate PU signals
pu_received_power = np.array([calculate_received_power(pt, d, alpha, pu_shadowing_std) 
                             for pt, d, alpha in zip(pu_transmit_power, pu_distance, pu_path_loss_exponent)])

# Generate PUEA signal data
puea_transmit_power = np.random.normal(puea_transmit_power_mean, puea_transmit_power_std, num_puea_samples)
puea_distance = np.random.normal(puea_distance_mean, puea_distance_std, num_puea_samples)
puea_path_loss_exponent = np.random.normal(puea_path_loss_exponent_mean, puea_path_loss_exponent_std, num_puea_samples)
puea_shadowing = np.random.normal(0, puea_shadowing_std, num_puea_samples)

# Calculate received power for PUEA signals
puea_received_power = np.array([calculate_received_power(pt, d, alpha, puea_shadowing_std) 
                               for pt, d, alpha in zip(puea_transmit_power, puea_distance, puea_path_loss_exponent)])

# Create dataframes for PU and PUEA signals
pu_df = pd.DataFrame({
    'transmit_power': pu_transmit_power,
    'distance': pu_distance,
    'path_loss_exponent': pu_path_loss_exponent,
    'shadowing_effect': pu_shadowing,
    'received_power': pu_received_power,
    'label': 0  # 0 for legitimate PU
})

puea_df = pd.DataFrame({
    'transmit_power': puea_transmit_power,
    'distance': puea_distance,
    'path_loss_exponent': puea_path_loss_exponent,
    'shadowing_effect': puea_shadowing,
    'received_power': puea_received_power,
    'label': 1  # 1 for PUEA
})

# Combine the dataframes
dataset = pd.concat([pu_df, puea_df], ignore_index=True)

# Shuffle the dataset
dataset = dataset.sample(frac=1).reset_index(drop=True)

# Save the dataset to a CSV file
dataset.to_csv('puea_detection_dataset.csv', index=False)

print(f"Dataset generated with {len(dataset)} samples")
print(f"PU signals: {num_pu_samples}, PUEA signals: {num_puea_samples}")
print(f"Features: {', '.join(dataset.columns[:-1])}")
\end{lstlisting}

\section{Feature Preprocessing}
Before applying clustering algorithms, we preprocess the dataset to ensure optimal performance. The preprocessing steps include:

\subsection{Feature Scaling}
Since the features in our dataset have different scales and units, we apply standardization to transform them to have zero mean and unit variance. This ensures that features with larger scales do not dominate the clustering process.

\begin{lstlisting}[language=Python, caption=Feature Scaling Code]
# Load the dataset
dataset = pd.read_csv('puea_detection_dataset.csv')

# Extract features and labels
X = dataset.drop('label', axis=1)
y_true = dataset['label']

# Standardize features
scaler = StandardScaler()
X_scaled = scaler.fit_transform(X)

# Convert back to DataFrame for easier handling
X_scaled_df = pd.DataFrame(X_scaled, columns=X.columns)
\end{lstlisting}

\subsection{Feature Selection}
Although our dataset has only five features, it is important to identify the most discriminative ones for clustering. We can use principal component analysis (PCA) for dimensionality reduction or feature importance analysis to select the most relevant features.

\begin{lstlisting}[language=Python, caption=Feature Selection Code]
from sklearn.decomposition import PCA

# Apply PCA for dimensionality reduction
pca = PCA(n_components=2)
X_pca = pca.fit_transform(X_scaled)

# Percentage of variance explained by each component
explained_variance = pca.explained_variance_ratio_
print(f"Explained variance ratio: {explained_variance}")
print(f"Total explained variance: {sum(explained_variance):.2f}")

# Plot PCA results to visualize data distribution
plt.figure(figsize=(10, 6))
scatter = plt.scatter(X_pca[:, 0], X_pca[:, 1], c=y_true, alpha=0.6, cmap='viridis')
plt.colorbar(scatter, label='Class (0: PU, 1: PUEA)')
plt.xlabel('Principal Component 1')
plt.ylabel('Principal Component 2')
plt.title('PCA of PUEA Detection Dataset')
plt.grid(alpha=0.3)
plt.savefig('pca_visualization.png', dpi=300, bbox_inches='tight')
plt.show()
\end{lstlisting}

\section{Clustering-Based Detection}
We apply three clustering algorithms—DBSCAN, Agglomerative Hierarchical Clustering, and K-means—to identify and separate legitimate PU signals from PUEA signals.

\subsection{K-means Clustering}
K-means is a partition-based clustering algorithm that divides the dataset into K distinct, non-overlapping clusters. For our PUEA detection problem, we set K=2 to separate the dataset into two clusters: one for legitimate PUs and one for PUEAs.

\begin{lstlisting}[language=Python, caption=K-means Clustering Code]
from sklearn.cluster import KMeans
from sklearn.metrics import confusion_matrix, accuracy_score, classification_report

# Apply K-means clustering
kmeans = KMeans(n_clusters=2, random_state=42, n_init=10)
kmeans_labels = kmeans.fit_predict(X_scaled)

# Map cluster labels to original classes (may need adjustment based on results)
# Since we know there should be two clusters (PU and PUEA)
# We need to check if the assigned labels match the true labels
if (kmeans_labels == y_true).mean() < 0.5:
    kmeans_labels = 1 - kmeans_labels

# Evaluate clustering performance
kmeans_accuracy = accuracy_score(y_true, kmeans_labels)
kmeans_conf_matrix = confusion_matrix(y_true, kmeans_labels)

print(f"K-means Accuracy: {kmeans_accuracy:.4f}")
print("K-means Confusion Matrix:")
print(kmeans_conf_matrix)
print("K-means Classification Report:")
print(classification_report(y_true, kmeans_labels))

# Visualize the clustering results
plt.figure(figsize=(10, 6))
scatter = plt.scatter(X_pca[:, 0], X_pca[:, 1], c=kmeans_labels, alpha=0.6, cmap='viridis')
plt.scatter(kmeans.cluster_centers_[:, 0], kmeans.cluster_centers_[:, 1], 
            c='red', marker='X', s=200, label='Cluster Centers')
plt.colorbar(scatter, label='Cluster Label')
plt.xlabel('Principal Component 1')
plt.ylabel('Principal Component 2')
plt.title('K-means Clustering Results')
plt.legend()
plt.grid(alpha=0.3)
plt.savefig('kmeans_clustering.png', dpi=300, bbox_inches='tight')
plt.show()
\end{lstlisting}

\subsection{DBSCAN Clustering}
DBSCAN (Density-Based Spatial Clustering of Applications with Noise) is a density-based clustering algorithm that groups together points that are closely packed while marking points in low-density regions as outliers. DBSCAN is particularly suitable for PUEA detection as it can identify anomalous signals without requiring the number of clusters to be specified in advance.

\begin{lstlisting}[language=Python, caption=DBSCAN Clustering Code]
from sklearn.cluster import DBSCAN

# Apply DBSCAN clustering
# The eps and min_samples parameters need to be tuned for optimal performance
dbscan = DBSCAN(eps=0.5, min_samples=5)
dbscan_labels = dbscan.fit_predict(X_scaled)

# DBSCAN labels outliers as -1, we need to map these properly
# Assuming outliers are PUEAs (label 1) and the main cluster is PUs (label 0)
dbscan_mapped_labels = np.where(dbscan_labels == -1, 1, 0)

# Check if we need to invert the labels
if (dbscan_mapped_labels == y_true).mean() < 0.5:
    dbscan_mapped_labels = 1 - dbscan_mapped_labels

# Evaluate clustering performance
dbscan_accuracy = accuracy_score(y_true, dbscan_mapped_labels)
dbscan_conf_matrix = confusion_matrix(y_true, dbscan_mapped_labels)

print(f"DBSCAN Accuracy: {dbscan_accuracy:.4f}")
print("DBSCAN Confusion Matrix:")
print(dbscan_conf_matrix)
print("DBSCAN Classification Report:")
print(classification_report(y_true, dbscan_mapped_labels))

# Visualize the clustering results
plt.figure(figsize=(10, 6))
scatter = plt.scatter(X_pca[:, 0], X_pca[:, 1], c=dbscan_mapped_labels, alpha=0.6, cmap='viridis')
plt.colorbar(scatter, label='Cluster Label')
plt.xlabel('Principal Component 1')
plt.ylabel('Principal Component 2')
plt.title('DBSCAN Clustering Results')
plt.grid(alpha=0.3)
plt.savefig('dbscan_clustering.png', dpi=300, bbox_inches='tight')
plt.show()
\end{lstlisting}

\subsection{Agglomerative Hierarchical Clustering}
Agglomerative Hierarchical Clustering starts with each data point as a singleton cluster and then progressively merges pairs of clusters until all points are in a single cluster. By cutting the dendrogram at an appropriate level, we can obtain the desired number of clusters (in our case, two).

\begin{lstlisting}[language=Python, caption=Agglomerative Clustering Code]
from sklearn.cluster import AgglomerativeClustering
from scipy.cluster.hierarchy import dendrogram, linkage

# Apply Agglomerative Clustering
agg_clustering = AgglomerativeClustering(n_clusters=2, linkage='ward')
agg_labels = agg_clustering.fit_predict(X_scaled)

# Check if we need to invert the labels
if (agg_labels == y_true).mean() < 0.5:
    agg_labels = 1 - agg_labels

# Evaluate clustering performance
agg_accuracy = accuracy_score(y_true, agg_labels)
agg_conf_matrix = confusion_matrix(y_true, agg_labels)

print(f"Agglomerative Clustering Accuracy: {agg_accuracy:.4f}")
print("Agglomerative Clustering Confusion Matrix:")
print(agg_conf_matrix)
print("Agglomerative Clustering Classification Report:")
print(classification_report(y_true, agg_labels))

# Create linkage matrix for dendrogram visualization
# We'll use only a subset of data for clearer visualization
subset_indices = np.random.choice(len(X_scaled), size=100, replace=False)
X_subset = X_scaled[subset_indices]
y_subset = y_true.iloc[subset_indices]

# Calculate linkage matrix
linkage_matrix = linkage(X_subset, method='ward')

# Plot dendrogram
plt.figure(figsize=(12, 8))
dendrogram(linkage_matrix)
plt.title('Hierarchical Clustering Dendrogram')
plt.xlabel('Sample Index')
plt.ylabel('Distance')
plt.savefig('hierarchical_dendrogram.png', dpi=300, bbox_inches='tight')
plt.show()

# Visualize the clustering results
plt.figure(figsize=(10, 6))
scatter = plt.scatter(X_pca[:, 0], X_pca[:, 1], c=agg_labels, alpha=0.6, cmap='viridis')
plt.colorbar(scatter, label='Cluster Label')
plt.xlabel('Principal Component 1')
plt.ylabel('Principal Component 2')
plt.title('Agglomerative Clustering Results')
plt.grid(alpha=0.3)
plt.savefig('agglomerative_clustering.png', dpi=300, bbox_inches='tight')
plt.show()
\end{lstlisting}

\section{Iterative Clustering Method}
To improve the detection accuracy, we propose an iterative clustering method that combines the strengths of different clustering algorithms. The basic idea is to apply one clustering algorithm first, use its results to identify potential PUEAs, and then apply another algorithm to refine the results.

\begin{lstlisting}[language=Python, caption=Iterative Clustering Method]
def iterative_clustering(X, initial_clustering_method='kmeans', refinement_method='dbscan'):
    """
    Perform iterative clustering for improved PUEA detection
    
    Parameters:
        X (array): Feature matrix
        initial_clustering_method (str): Initial clustering method ('kmeans', 'dbscan', or 'agglomerative')
        refinement_method (str): Method for refinement ('kmeans', 'dbscan', or 'agglomerative')
    
    Returns:
        labels (array): Final cluster assignments
    """
    # Step 1: Initial clustering
    if initial_clustering_method == 'kmeans':
        initial_model = KMeans(n_clusters=2, random_state=42, n_init=10)
        initial_labels = initial_model.fit_predict(X)
    elif initial_clustering_method == 'dbscan':
        initial_model = DBSCAN(eps=0.5, min_samples=5)
        initial_labels = initial_model.fit_predict(X)
        # Map DBSCAN outliers (-1) to label 1
        initial_labels = np.where(initial_labels == -1, 1, 0)
    elif initial_clustering_method == 'agglomerative':
        initial_model = AgglomerativeClustering(n_clusters=2, linkage='ward')
        initial_labels = initial_model.fit_predict(X)
    
    # Step 2: Identify potential outliers
    # Assuming smaller cluster is more likely to be PUEA
    if np.sum(initial_labels == 0) < np.sum(initial_labels == 1):
        potential_puea_indices = np.where(initial_labels == 0)[0]
        potential_pu_indices = np.where(initial_labels == 1)[0]
    else:
        potential_puea_indices = np.where(initial_labels == 1)[0]
        potential_pu_indices = np.where(initial_labels == 0)[0]
    
    # Step 3: Refinement using another clustering method
    final_labels = np.zeros(len(X))
    
    # Mark PU cluster from initial clustering
    final_labels[potential_pu_indices] = 0
    
    # Apply refinement clustering only on potential PUEA samples
    X_potential_puea = X[potential_puea_indices]
    
    if len(X_potential_puea) > 0:
        if refinement_method == 'kmeans':
            refinement_model = KMeans(n_clusters=2, random_state=42, n_init=10)
        elif refinement_method == 'dbscan':
            refinement_model = DBSCAN(eps=0.4, min_samples=3)  # Tighter parameters for refinement
        elif refinement_method == 'agglomerative':
            refinement_model = AgglomerativeClustering(n_clusters=2, linkage='ward')
        
        refinement_labels = refinement_model.fit_predict(X_potential_puea)
        
        # Map refinement labels and assign to final labels
        if refinement_method == 'dbscan':
            # DBSCAN outliers (-1) are definitely PUEAs
            final_labels[potential_puea_indices[refinement_labels == -1]] = 1
            
            # For non-outliers, pick the smaller cluster as PUEA (more conservative approach)
            non_outlier_indices = potential_puea_indices[refinement_labels != -1]
            if len(non_outlier_indices) > 0:
                non_outlier_labels = refinement_labels[refinement_labels != -1]
                if np.sum(non_outlier_labels == 0) < np.sum(non_outlier_labels == 1):
                    final_labels[potential_puea_indices[refinement_labels == 0]] = 1
                else:
                    final_labels[potential_puea_indices[refinement_labels == 1]] = 1
        else:
            # For K-means and Agglomerative, pick the smaller cluster as PUEA
            if np.sum(refinement_labels == 0) < np.sum(refinement_labels == 1):
                final_labels[potential_puea_indices[refinement_labels == 0]] = 1
            else:
                final_labels[potential_puea_indices[refinement_labels == 1]] = 1
    
    return final_labels

# Apply iterative clustering
iterative_labels = iterative_clustering(X_scaled, 'kmeans', 'dbscan')

# Evaluate performance
iterative_accuracy = accuracy_score(y_true, iterative_labels)
iterative_conf_matrix = confusion_matrix(y_true, iterative_labels)

print(f"Iterative Clustering Accuracy: {iterative_accuracy:.4f}")
print("Iterative Clustering Confusion Matrix:")
print(iterative_conf_matrix)
print("Iterative Clustering Classification Report:")
print(classification_report(y_true, iterative_labels))

# Visualize the clustering results
plt.figure(figsize=(10, 6))
scatter = plt.scatter(X_pca[:, 0], X_pca[:, 1], c=iterative_labels, alpha=0.6, cmap='viridis')
plt.colorbar(scatter, label='Cluster Label')
plt.xlabel('Principal Component 1')
plt.ylabel('Principal Component 2')
plt.title('Iterative Clustering Results')
plt.grid(alpha=0.3)
plt.savefig('iterative_clustering.png', dpi=300, bbox_inches='tight')
plt.show()
\end{lstlisting}

\section{Performance Evaluation Metrics}
To evaluate the performance of the different clustering methods, we use the following metrics:

\begin{enumerate}
    \item \textbf{Accuracy}: The proportion of correctly classified instances.
    \item \textbf{Precision}: The proportion of correctly identified PUEAs among all instances classified as PUEAs.
    \item \textbf{Recall}: The proportion of correctly identified PUEAs among all actual PUEA instances.
    \item \textbf{F1-score}: The harmonic mean of precision and recall.
    \item \textbf{Confusion Matrix}: Shows the number of true positives, false positives, true negatives, and false negatives.
\end{enumerate}

\begin{lstlisting}[language=Python, caption=Performance Comparison Code]
import matplotlib.pyplot as plt
import numpy as np
from sklearn.metrics import accuracy_score, precision_score, recall_score, f1_score

# Collect performance metrics for all methods
methods = ['K-means', 'DBSCAN', 'Agglomerative', 'Iterative']
labels = [kmeans_labels, dbscan_mapped_labels, agg_labels, iterative_labels]

accuracies = [accuracy_score(y_true, label) for label in labels]
precisions = [precision_score(y_true, label) for label in labels]
recalls = [recall_score(y_true, label) for label in labels]
f1_scores = [f1_score(y_true, label) for label in labels]

# Plot performance comparison
metrics = ['Accuracy', 'Precision', 'Recall', 'F1-score']
values = np.array([accuracies, precisions, recalls, f1_scores])

x = np.arange(len(metrics))
width = 0.2
multiplier = 0

fig, ax = plt.subplots(figsize=(12, 8))

for i, method in enumerate(methods):
    offset = width * multiplier
    rects = ax.bar(x + offset, values[:, i], width, label=method)
    multiplier += 1

ax.set_ylabel('Score')
ax.set_title('Performance Comparison of Clustering Methods')
ax.set_xticks(x + width * (len(methods) - 1) / 2)
ax.set_xticklabels(metrics)
ax.legend(loc='upper center', bbox_to_anchor=(0.5, -0.05), ncol=len(methods))
ax.set_ylim(0, 1.1)

# Add value labels on bars
for i in range(len(metrics)):
    for j in range(len(methods)):
        ax.text(i + width * j - width/2, values[i, j] + 0.01, 
                f'{values[i, j]:.2f}', ha='center', va='bottom', fontsize=8)

plt.grid(axis='y', alpha=0.3)
plt.tight_layout()
plt.savefig('performance_comparison.png', dpi=300, bbox_inches='tight')
plt.show()
\end{lstlisting}

\section{Summary}
In this chapter, we presented a comprehensive methodology for detecting Primary User Emulation Attacks in Cognitive Radio Networks. Our approach begins with generating a realistic dataset that captures various signal propagation factors affecting both legitimate PU and PUEA signals. We then preprocess the data using standardization and optionally apply dimensionality reduction techniques.

For detection, we explored three clustering algorithms—K-means, DBSCAN, and Agglomerative Hierarchical Clustering—each with its strengths and limitations. To leverage the advantages of different algorithms, we proposed an iterative clustering method that combines multiple techniques for improved detection accuracy.

The performance evaluation metrics enable us to compare the effectiveness of different clustering approaches and determine which method is most suitable for PUEA detection in various scenarios. The methodology described in this chapter provides a foundation for implementing and evaluating the proposed PUEA detection system, which will be presented in the next chapter.

% Add references section
\begin{thebibliography}{99}
\bibitem{rappaport1996wireless} Rappaport, T. S. (1996). Wireless communications: principles and practice. Prentice Hall.
\end{thebibliography}

\chapter{Implementation and Experimental Setup}

\section{Implementation Overview}
This chapter describes the implementation details and experimental setup for evaluating our proposed approach for Primary User Emulation Attack (PUEA) detection in Cognitive Radio Networks. We provide a comprehensive description of the software tools, libraries, and the experimental environment used for implementing and testing the proposed clustering-based detection methods.

\section{Software Environment}
The implementation of our proposed approach is carried out using Python, a widely used programming language for data analysis and machine learning applications. Python offers rich libraries for scientific computing, data processing, and machine learning, making it suitable for our research.

\subsection{Python Libraries}
The following key libraries were used in our implementation:

\begin{itemize}
    \item \textbf{NumPy}: For numerical computation and handling multi-dimensional arrays.
    \item \textbf{Pandas}: For data manipulation and analysis.
    \item \textbf{Scikit-learn}: For machine learning algorithms implementation, including clustering algorithms and evaluation metrics.
    \item \textbf{Matplotlib} and \textbf{Seaborn}: For data visualization and plotting.
    \item \textbf{SciPy}: For scientific computing, especially for hierarchical clustering and additional statistical functions.
\end{itemize}

\subsection{Development Environment}
The development and experimentation were performed on the following hardware and software configuration:
\begin{itemize}
    \item \textbf{Hardware}: Laptop with Intel Core i7-10750H CPU @ 2.60GHz, 16GB RAM
    \item \textbf{Operating System}: Windows 10 64-bit
    \item \textbf{Python Version}: Python 3.8.5
    \item \textbf{IDE}: Jupyter Notebook and Visual Studio Code
\end{itemize}

\section{Dataset Generation Implementation}
As described in the methodology chapter, we implemented a synthetic dataset generation process to create a realistic dataset for evaluating our PUEA detection approach. Here, we provide additional implementation details and the exact configuration parameters used in our experiments.

\subsection{Configuration Parameters}
The dataset was generated with the following configuration parameters:

\begin{table}[h]
    \centering
    \caption{Dataset Configuration Parameters}
    \label{tab:dataset_params}
    \begin{tabular}{lcc}
        \toprule
        Parameter & Primary User (PU) & PUEA \\
        \midrule
        Number of samples & 300 & 150 \\
        Transmit power mean (dBm) & 30 & 35 \\
        Transmit power standard deviation (dBm) & 1 & 2 \\
        Distance mean (meters) & 500 & 450 \\
        Distance standard deviation (meters) & 100 & 120 \\
        Path loss exponent mean & 3.0 & 2.8 \\
        Path loss exponent standard deviation & 0.2 & 0.3 \\
        Shadowing standard deviation (dB) & 4 & 5 \\
        \bottomrule
    \end{tabular}
\end{table}

\subsection{Implementation Process}
The dataset generation involves the following steps:

\begin{enumerate}
    \item Setting random seed for reproducibility
    \item Defining the signal propagation model based on the log-normal shadowing path loss model
    \item Generating parameters for PU signals based on the configuration
    \item Generating parameters for PUEA signals based on the configuration
    \item Calculating received power using the signal propagation model
    \item Combining PU and PUEA signals into a single dataset
    \item Labeling the dataset (0 for PU, 1 for PUEA)
    \item Shuffling the dataset to ensure random ordering
    \item Saving the dataset to a CSV file for further processing
\end{enumerate}

\section{Data Preprocessing Implementation}
Before applying clustering algorithms, we preprocess the dataset to ensure optimal performance. The preprocessing steps are implemented as follows:

\subsection{Data Loading and Exploration}
First, we load the dataset from the CSV file and explore its basic statistics to understand the data distribution:

\begin{lstlisting}[language=Python, caption=Data Exploration Code]
import pandas as pd
import numpy as np
import matplotlib.pyplot as plt
import seaborn as sns

# Load the dataset
dataset = pd.read_csv('puea_detection_dataset.csv')

# Display basic statistics
print(dataset.describe())

# Count of PU and PUEA samples
print(f"PU samples: {len(dataset[dataset['label'] == 0])}")
print(f"PUEA samples: {len(dataset[dataset['label'] == 1])}")

# Plot histograms for each feature to visualize distributions
fig, axes = plt.subplots(2, 3, figsize=(18, 10))
axes = axes.flatten()

features = ['transmit_power', 'distance', 'path_loss_exponent', 
            'shadowing_effect', 'received_power']

for i, feature in enumerate(features):
    sns.histplot(data=dataset, x=feature, hue='label', bins=30, kde=True, ax=axes[i])
    axes[i].set_title(f'{feature} Distribution')
    axes[i].legend(['PU', 'PUEA'])

plt.tight_layout()
plt.savefig('feature_distributions.png', dpi=300, bbox_inches='tight')
plt.show()

# Plot correlation matrix
plt.figure(figsize=(10, 8))
correlation = dataset.corr()
sns.heatmap(correlation, annot=True, cmap='coolwarm', fmt='.2f')
plt.title('Feature Correlation Matrix')
plt.tight_layout()
plt.savefig('correlation_matrix.png', dpi=300, bbox_inches='tight')
plt.show()
\end{lstlisting}

\subsection{Feature Scaling Implementation}
We implement feature scaling using the StandardScaler from scikit-learn, which transforms features to have zero mean and unit variance:

\begin{lstlisting}[language=Python, caption=Feature Scaling Implementation]
from sklearn.preprocessing import StandardScaler

# Extract features and labels
X = dataset.drop('label', axis=1)
y_true = dataset['label']

# Standardize features
scaler = StandardScaler()
X_scaled = scaler.fit_transform(X)

# Convert back to DataFrame for easier handling
X_scaled_df = pd.DataFrame(X_scaled, columns=X.columns)

# Display basic statistics after scaling
print(X_scaled_df.describe())
\end{lstlisting}

\subsection{Feature Selection and Dimensionality Reduction}
For better visualization and potentially improved clustering performance, we implement PCA to reduce the dimensionality of the data:

\begin{lstlisting}[language=Python, caption=PCA Implementation]
from sklearn.decomposition import PCA

# Apply PCA for dimensionality reduction
pca = PCA(n_components=2)
X_pca = pca.fit_transform(X_scaled)

# Percentage of variance explained by each component
explained_variance = pca.explained_variance_ratio_
print(f"Explained variance ratio: {explained_variance}")
print(f"Total explained variance: {sum(explained_variance):.2f}")

# Feature importance in principal components
for i, component in enumerate(pca.components_):
    print(f"Principal Component {i+1}:")
    for j, feature in enumerate(X.columns):
        print(f"{feature}: {component[j]:.3f}")
    print()

# Plot PCA results
plt.figure(figsize=(10, 6))
scatter = plt.scatter(X_pca[:, 0], X_pca[:, 1], c=y_true, alpha=0.6, cmap='viridis')
plt.colorbar(scatter, label='Class (0: PU, 1: PUEA)')
plt.xlabel('Principal Component 1')
plt.ylabel('Principal Component 2')
plt.title('PCA of PUEA Detection Dataset')
plt.grid(alpha=0.3)
plt.savefig('pca_visualization.png', dpi=300, bbox_inches='tight')
plt.show()
\end{lstlisting}

\section{Clustering Algorithm Implementation}
We implement three clustering algorithms for PUEA detection: K-means, DBSCAN, and Agglomerative Hierarchical Clustering. Each algorithm is configured with appropriate parameters based on the nature of our dataset.

\subsection{K-means Implementation}
For K-means, we set the number of clusters to 2 since we aim to separate PU and PUEA signals:

\begin{lstlisting}[language=Python, caption=K-means Implementation and Parameter Tuning]
from sklearn.cluster import KMeans
from sklearn.metrics import confusion_matrix, accuracy_score, classification_report

# Function to evaluate K-means with different parameters
def evaluate_kmeans(X, y_true, n_init_values=[10, 20, 30], random_states=[42, 100, 200]):
    results = []
    
    for n_init in n_init_values:
        for rs in random_states:
            # Apply K-means clustering
            kmeans = KMeans(n_clusters=2, random_state=rs, n_init=n_init)
            kmeans_labels = kmeans.fit_predict(X)
            
            # Map cluster labels to original classes
            if (kmeans_labels == y_true).mean() < 0.5:
                kmeans_labels = 1 - kmeans_labels
                
            # Calculate accuracy
            accuracy = accuracy_score(y_true, kmeans_labels)
            
            results.append({
                'n_init': n_init,
                'random_state': rs,
                'accuracy': accuracy
            })
    
    return pd.DataFrame(results)

# Evaluate K-means with different parameters
kmeans_results = evaluate_kmeans(X_scaled, y_true)
print(kmeans_results.sort_values('accuracy', ascending=False).head())

# Apply K-means with the best parameters
best_params = kmeans_results.sort_values('accuracy', ascending=False).iloc[0]
kmeans = KMeans(n_clusters=2, random_state=int(best_params['random_state']), 
                n_init=int(best_params['n_init']))
kmeans_labels = kmeans.fit_predict(X_scaled)

# Map cluster labels to original classes
if (kmeans_labels == y_true).mean() < 0.5:
    kmeans_labels = 1 - kmeans_labels

# Evaluate clustering performance
kmeans_accuracy = accuracy_score(y_true, kmeans_labels)
kmeans_conf_matrix = confusion_matrix(y_true, kmeans_labels)

print(f"K-means Accuracy: {kmeans_accuracy:.4f}")
print("K-means Confusion Matrix:")
print(kmeans_conf_matrix)
print("K-means Classification Report:")
print(classification_report(y_true, kmeans_labels))

# Visualize the clustering results
plt.figure(figsize=(10, 6))
scatter = plt.scatter(X_pca[:, 0], X_pca[:, 1], c=kmeans_labels, alpha=0.6, cmap='viridis')
plt.scatter(kmeans.cluster_centers_[:, 0], kmeans.cluster_centers_[:, 1], 
            c='red', marker='X', s=200, label='Cluster Centers')
plt.colorbar(scatter, label='Cluster Label')
plt.xlabel('Principal Component 1')
plt.ylabel('Principal Component 2')
plt.title('K-means Clustering Results')
plt.legend()
plt.grid(alpha=0.3)
plt.savefig('kmeans_clustering.png', dpi=300, bbox_inches='tight')
plt.show()
\end{lstlisting}

\subsection{DBSCAN Implementation}
For DBSCAN, we need to determine appropriate values for the epsilon (eps) and minimum samples (min\_samples) parameters:

\begin{lstlisting}[language=Python, caption=DBSCAN Implementation and Parameter Tuning]
from sklearn.cluster import DBSCAN
from sklearn.neighbors import NearestNeighbors

# Find optimal epsilon value using the k-distance graph
def find_optimal_eps(X, k=5):
    neigh = NearestNeighbors(n_neighbors=k)
    nbrs = neigh.fit(X)
    distances, indices = nbrs.kneighbors(X)
    
    # Sort distances in ascending order
    distances = np.sort(distances[:, -1])
    
    # Plot k-distance graph
    plt.figure(figsize=(10, 6))
    plt.plot(distances)
    plt.xlabel('Data Points (sorted)')
    plt.ylabel(f'Distance to {k}th Nearest Neighbor')
    plt.title('K-distance Graph for Determining Optimal Epsilon')
    plt.grid(True)
    plt.savefig('k_distance_graph.png', dpi=300, bbox_inches='tight')
    plt.show()
    
    return distances

# Find optimal epsilon
distances = find_optimal_eps(X_scaled, k=5)

# Function to evaluate DBSCAN with different parameters
def evaluate_dbscan(X, y_true, eps_values=[0.3, 0.4, 0.5, 0.6], min_samples_values=[3, 5, 7, 10]):
    results = []
    
    for eps in eps_values:
        for min_samples in min_samples_values:
            # Apply DBSCAN clustering
            dbscan = DBSCAN(eps=eps, min_samples=min_samples)
            dbscan_labels = dbscan.fit_predict(X)
            
            # DBSCAN labels outliers as -1, we need to map these properly
            dbscan_mapped_labels = np.where(dbscan_labels == -1, 1, 0)
            
            # Check if we need to invert the labels
            if (dbscan_mapped_labels == y_true).mean() < 0.5:
                dbscan_mapped_labels = 1 - dbscan_mapped_labels
                
            # Calculate accuracy
            accuracy = accuracy_score(y_true, dbscan_mapped_labels)
            
            # Calculate number of clusters (excluding noise)
            n_clusters = len(set(dbscan_labels)) - (1 if -1 in dbscan_labels else 0)
            
            results.append({
                'eps': eps,
                'min_samples': min_samples,
                'n_clusters': n_clusters,
                'noise_points': (dbscan_labels == -1).sum(),
                'accuracy': accuracy
            })
    
    return pd.DataFrame(results)

# Evaluate DBSCAN with different parameters
dbscan_results = evaluate_dbscan(X_scaled, y_true)
print(dbscan_results.sort_values('accuracy', ascending=False).head())

# Apply DBSCAN with the best parameters
best_params = dbscan_results.sort_values('accuracy', ascending=False).iloc[0]
dbscan = DBSCAN(eps=best_params['eps'], min_samples=int(best_params['min_samples']))
dbscan_labels = dbscan.fit_predict(X_scaled)

# Map DBSCAN labels
dbscan_mapped_labels = np.where(dbscan_labels == -1, 1, 0)
if (dbscan_mapped_labels == y_true).mean() < 0.5:
    dbscan_mapped_labels = 1 - dbscan_mapped_labels

# Evaluate clustering performance
dbscan_accuracy = accuracy_score(y_true, dbscan_mapped_labels)
dbscan_conf_matrix = confusion_matrix(y_true, dbscan_mapped_labels)

print(f"DBSCAN Accuracy: {dbscan_accuracy:.4f}")
print("DBSCAN Confusion Matrix:")
print(dbscan_conf_matrix)
print("DBSCAN Classification Report:")
print(classification_report(y_true, dbscan_mapped_labels))

# Visualize the clustering results
plt.figure(figsize=(10, 6))
scatter = plt.scatter(X_pca[:, 0], X_pca[:, 1], c=dbscan_mapped_labels, alpha=0.6, cmap='viridis')
plt.colorbar(scatter, label='Cluster Label')
plt.xlabel('Principal Component 1')
plt.ylabel('Principal Component 2')
plt.title('DBSCAN Clustering Results')
plt.grid(alpha=0.3)
plt.savefig('dbscan_clustering.png', dpi=300, bbox_inches='tight')
plt.show()
\end{lstlisting}

\subsection{Agglomerative Hierarchical Clustering Implementation}
For Agglomerative Hierarchical Clustering, we experiment with different linkage criteria:

\begin{lstlisting}[language=Python, caption=Agglomerative Clustering Implementation and Parameter Tuning]
from sklearn.cluster import AgglomerativeClustering
from scipy.cluster.hierarchy import dendrogram, linkage

# Function to evaluate Agglomerative Clustering with different parameters
def evaluate_agglomerative(X, y_true, linkage_methods=['ward', 'complete', 'average', 'single']):
    results = []
    
    for method in linkage_methods:
        # Apply Agglomerative Clustering
        agg_clustering = AgglomerativeClustering(n_clusters=2, linkage=method)
        agg_labels = agg_clustering.fit_predict(X)
        
        # Check if we need to invert the labels
        if (agg_labels == y_true).mean() < 0.5:
            agg_labels = 1 - agg_labels
            
        # Calculate accuracy
        accuracy = accuracy_score(y_true, agg_labels)
        
        results.append({
            'linkage_method': method,
            'accuracy': accuracy
        })
    
    return pd.DataFrame(results)

# Evaluate Agglomerative Clustering with different parameters
agg_results = evaluate_agglomerative(X_scaled, y_true)
print(agg_results.sort_values('accuracy', ascending=False))

# Apply Agglomerative Clustering with the best parameters
best_method = agg_results.sort_values('accuracy', ascending=False).iloc[0]['linkage_method']
agg_clustering = AgglomerativeClustering(n_clusters=2, linkage=best_method)
agg_labels = agg_clustering.fit_predict(X_scaled)

# Check if we need to invert the labels
if (agg_labels == y_true).mean() < 0.5:
    agg_labels = 1 - agg_labels

# Evaluate clustering performance
agg_accuracy = accuracy_score(y_true, agg_labels)
agg_conf_matrix = confusion_matrix(y_true, agg_labels)

print(f"Agglomerative Clustering Accuracy: {agg_accuracy:.4f}")
print("Agglomerative Clustering Confusion Matrix:")
print(agg_conf_matrix)
print("Agglomerative Clustering Classification Report:")
print(classification_report(y_true, agg_labels))

# Create linkage matrix for dendrogram visualization
# We'll use only a subset of data for clearer visualization
subset_indices = np.random.choice(len(X_scaled), size=100, replace=False)
X_subset = X_scaled[subset_indices]
y_subset = y_true.iloc[subset_indices]

# Calculate linkage matrix
linkage_matrix = linkage(X_subset, method=best_method)

# Plot dendrogram
plt.figure(figsize=(12, 8))
dendrogram(linkage_matrix)
plt.title(f'Hierarchical Clustering Dendrogram ({best_method} Linkage)')
plt.xlabel('Sample Index')
plt.ylabel('Distance')
plt.axhline(y=0.8, c='k', linestyle='--', alpha=0.3)
plt.text(10, 0.85, 'Cut for 2 clusters', fontsize=12)
plt.savefig('hierarchical_dendrogram.png', dpi=300, bbox_inches='tight')
plt.show()

# Visualize the clustering results
plt.figure(figsize=(10, 6))
scatter = plt.scatter(X_pca[:, 0], X_pca[:, 1], c=agg_labels, alpha=0.6, cmap='viridis')
plt.colorbar(scatter, label='Cluster Label')
plt.xlabel('Principal Component 1')
plt.ylabel('Principal Component 2')
plt.title(f'Agglomerative Clustering Results ({best_method} Linkage)')
plt.grid(alpha=0.3)
plt.savefig('agglomerative_clustering.png', dpi=300, bbox_inches='tight')
plt.show()
\end{lstlisting}

\section{Iterative Clustering Implementation}
We implement our proposed iterative clustering approach by combining multiple clustering algorithms:

\begin{lstlisting}[language=Python, caption=Iterative Clustering Implementation]
def iterative_clustering(X, y_true=None, initial_method='kmeans', refinement_method='dbscan'):
    """
    Perform iterative clustering for improved PUEA detection
    
    Parameters:
        X (array): Feature matrix
        y_true (array, optional): True labels for performance evaluation
        initial_method (str): Initial clustering method ('kmeans', 'dbscan', or 'agglomerative')
        refinement_method (str): Method for refinement ('kmeans', 'dbscan', or 'agglomerative')
    
    Returns:
        tuple: (final_labels, performance_metrics)
    """
    # Step 1: Initial clustering
    if initial_method == 'kmeans':
        initial_model = KMeans(n_clusters=2, random_state=42, n_init=10)
        initial_labels = initial_model.fit_predict(X)
    elif initial_method == 'dbscan':
        initial_model = DBSCAN(eps=0.5, min_samples=5)
        initial_labels = initial_model.fit_predict(X)
        # Map DBSCAN outliers (-1) to label 1 (assuming outliers are PUEAs)
        initial_labels = np.where(initial_labels == -1, 1, 0)
    elif initial_method == 'agglomerative':
        initial_model = AgglomerativeClustering(n_clusters=2, linkage='ward')
        initial_labels = initial_model.fit_predict(X)
    
    # If y_true is provided, map initial labels to original classes
    if y_true is not None:
        if (initial_labels == y_true).mean() < 0.5:
            initial_labels = 1 - initial_labels
    
    # Step 2: Identify potential outliers
    # Assuming smaller cluster is more likely to be PUEA
    if np.sum(initial_labels == 0) < np.sum(initial_labels == 1):
        potential_puea_indices = np.where(initial_labels == 0)[0]
        potential_pu_indices = np.where(initial_labels == 1)[0]
    else:
        potential_puea_indices = np.where(initial_labels == 1)[0]
        potential_pu_indices = np.where(initial_labels == 0)[0]
    
    # Step 3: Refinement using another clustering method
    final_labels = np.zeros(len(X))
    
    # Mark PU cluster from initial clustering
    final_labels[potential_pu_indices] = 0
    
    # Apply refinement clustering only on potential PUEA samples
    X_potential_puea = X[potential_puea_indices]
    
    if len(X_potential_puea) > 1:  # Need at least 2 samples for clustering
        if refinement_method == 'kmeans':
            refinement_model = KMeans(n_clusters=2, random_state=42, n_init=10)
        elif refinement_method == 'dbscan':
            refinement_model = DBSCAN(eps=0.4, min_samples=3)  # Tighter parameters for refinement
        elif refinement_method == 'agglomerative':
            refinement_model = AgglomerativeClustering(n_clusters=2, linkage='ward')
        
        refinement_labels = refinement_model.fit_predict(X_potential_puea)
        
        # Map refinement labels and assign to final labels
        if refinement_method == 'dbscan':
            # DBSCAN outliers (-1) are definitely PUEAs
            final_labels[potential_puea_indices[refinement_labels == -1]] = 1
            
            # For non-outliers, pick the smaller cluster as PUEA (more conservative approach)
            non_outlier_indices = np.where(refinement_labels != -1)[0]
            if len(non_outlier_indices) > 0:
                non_outlier_labels = refinement_labels[non_outlier_indices]
                if np.sum(non_outlier_labels == 0) < np.sum(non_outlier_labels == 1):
                    final_labels[potential_puea_indices[non_outlier_indices[non_outlier_labels == 0]]] = 1
                else:
                    final_labels[potential_puea_indices[non_outlier_indices[non_outlier_labels == 1]]] = 1
        else:
            # For K-means and Agglomerative, pick the smaller cluster as PUEA
            if np.sum(refinement_labels == 0) < np.sum(refinement_labels == 1):
                final_labels[potential_puea_indices[refinement_labels == 0]] = 1
            else:
                final_labels[potential_puea_indices[refinement_labels == 1]] = 1
    else:
        # If there's only one potential PUEA sample, mark it as PUEA
        final_labels[potential_puea_indices] = 1
    
    # Performance metrics
    metrics = {}
    if y_true is not None:
        # Check if we need to invert the labels
        if (final_labels == y_true).mean() < 0.5:
            final_labels = 1 - final_labels
            
        metrics['accuracy'] = accuracy_score(y_true, final_labels)
        metrics['precision'] = precision_score(y_true, final_labels)
        metrics['recall'] = recall_score(y_true, final_labels)
        metrics['f1'] = f1_score(y_true, final_labels)
        metrics['confusion_matrix'] = confusion_matrix(y_true, final_labels)
    
    return final_labels, metrics

# Test different combinations of initial and refinement methods
combinations = [
    ('kmeans', 'dbscan'),
    ('kmeans', 'agglomerative'),
    ('dbscan', 'kmeans'),
    ('dbscan', 'agglomerative'),
    ('agglomerative', 'kmeans'),
    ('agglomerative', 'dbscan')
]

combination_results = []

for init_method, refine_method in combinations:
    labels, metrics = iterative_clustering(X_scaled, y_true, init_method, refine_method)
    
    combination_results.append({
        'initial_method': init_method,
        'refinement_method': refine_method,
        'accuracy': metrics['accuracy'],
        'precision': metrics['precision'],
        'recall': metrics['recall'],
        'f1': metrics['f1']
    })

# Display results of all combinations
combination_df = pd.DataFrame(combination_results)
print(combination_df.sort_values('accuracy', ascending=False))

# Select the best combination
best_combination = combination_df.sort_values('accuracy', ascending=False).iloc[0]
best_init_method = best_combination['initial_method']
best_refine_method = best_combination['refinement_method']

# Apply iterative clustering with the best combination
iterative_labels, iterative_metrics = iterative_clustering(
    X_scaled, y_true, best_init_method, best_refine_method
)

print(f"\nBest Iterative Clustering Combination: {best_init_method} + {best_refine_method}")
print(f"Accuracy: {iterative_metrics['accuracy']:.4f}")
print(f"Precision: {iterative_metrics['precision']:.4f}")
print(f"Recall: {iterative_metrics['recall']:.4f}")
print(f"F1 Score: {iterative_metrics['f1']:.4f}")
print("Confusion Matrix:")
print(iterative_metrics['confusion_matrix'])
print("Classification Report:")
print(classification_report(y_true, iterative_labels))

# Visualize the clustering results
plt.figure(figsize=(10, 6))
scatter = plt.scatter(X_pca[:, 0], X_pca[:, 1], c=iterative_labels, alpha=0.6, cmap='viridis')
plt.colorbar(scatter, label='Cluster Label')
plt.xlabel('Principal Component 1')
plt.ylabel('Principal Component 2')
plt.title(f'Iterative Clustering Results ({best_init_method} + {best_refine_method})')
plt.grid(alpha=0.3)
plt.savefig('iterative_clustering.png', dpi=300, bbox_inches='tight')
plt.show()
\end{lstlisting}

\section{Performance Evaluation Implementation}
We implement comprehensive performance evaluation metrics to compare different clustering algorithms:

\begin{lstlisting}[language=Python, caption=Performance Evaluation Implementation]
def compare_clustering_methods(methods, labels, y_true):
    """
    Compare different clustering methods using various metrics.
    
    Parameters:
        methods (list): List of method names
        labels (list): List of cluster labels for each method
        y_true (array): True labels for comparison
    
    Returns:
        DataFrame: Performance metrics for each method
    """
    results = []
    
    for method, label in zip(methods, labels):
        metrics = {
            'method': method,
            'accuracy': accuracy_score(y_true, label),
            'precision': precision_score(y_true, label),
            'recall': recall_score(y_true, label),
            'f1': f1_score(y_true, label),
            'tn': confusion_matrix(y_true, label)[0, 0],
            'fp': confusion_matrix(y_true, label)[0, 1],
            'fn': confusion_matrix(y_true, label)[1, 0],
            'tp': confusion_matrix(y_true, label)[1, 1]
        }
        results.append(metrics)
    
    return pd.DataFrame(results)

# Compare all clustering methods
methods = ['K-means', 'DBSCAN', 'Agglomerative', 'Iterative']
all_labels = [kmeans_labels, dbscan_mapped_labels, agg_labels, iterative_labels]

comparison_results = compare_clustering_methods(methods, all_labels, y_true)
print(comparison_results)

# Plot performance comparison
plt.figure(figsize=(14, 8))

# Plot accuracy, precision, recall, f1
metrics = ['accuracy', 'precision', 'recall', 'f1']
x = np.arange(len(metrics))
width = 0.2
multiplier = 0

for i, method in enumerate(methods):
    offset = width * multiplier
    plt.bar(x + offset, comparison_results.loc[i, metrics], width, label=method)
    multiplier += 1

plt.axhline(y=0.5, color='r', linestyle='--', alpha=0.3, label='Random Classifier')
plt.ylabel('Score')
plt.title('Performance Comparison of Clustering Methods')
plt.xticks(x + width * (len(methods) - 1) / 2, metrics)
plt.legend(loc='upper center', bbox_to_anchor=(0.5, -0.05), ncol=len(methods))
plt.ylim(0, 1.1)

# Add value labels on bars
for i, metric in enumerate(metrics):
    for j, method in enumerate(methods):
        value = comparison_results.loc[j, metric]
        plt.text(i + width * j - width/2, value + 0.01, f'{value:.2f}', ha='center', va='bottom', fontsize=8)

plt.grid(axis='y', alpha=0.3)
plt.tight_layout()
plt.savefig('performance_comparison.png', dpi=300, bbox_inches='tight')
plt.show()

# Plot confusion matrix for the best method (iterative clustering)
def plot_confusion_matrix(cm, classes, title):
    """
    Plot confusion matrix with normalized values.
    """
    plt.figure(figsize=(8, 6))
    plt.imshow(cm, interpolation='nearest', cmap=plt.cm.Blues)
    plt.title(title)
    plt.colorbar()
    tick_marks = np.arange(len(classes))
    plt.xticks(tick_marks, classes)
    plt.yticks(tick_marks, classes)

    # Add text annotations
    thresh = cm.max() / 2.
    for i, j in itertools.product(range(cm.shape[0]), range(cm.shape[1])):
        plt.text(j, i, f'{cm[i, j]}',
                 horizontalalignment="center",
                 color="white" if cm[i, j] > thresh else "black")

    plt.ylabel('True label')
    plt.xlabel('Predicted label')
    plt.tight_layout()

# Import itertools for product function
import itertools

# Confusion matrix for the best method
best_method = comparison_results.sort_values('accuracy', ascending=False).iloc[0]['method']
best_index = methods.index(best_method)
best_cm = confusion_matrix(y_true, all_labels[best_index])

plot_confusion_matrix(best_cm, classes=['PU', 'PUEA'], 
                     title=f'Confusion Matrix - {best_method}')
plt.savefig('best_confusion_matrix.png', dpi=300, bbox_inches='tight')
plt.show()
\end{lstlisting}

\section{Experimental Scenarios}
To thoroughly evaluate our proposed approach, we conducted experiments under various scenarios to assess the robustness and effectiveness of the clustering-based PUEA detection methods.

\subsection{Varying Signal-to-Noise Ratio (SNR)}
We tested the performance of our detection methods under different signal-to-noise ratio conditions by adding varying levels of Gaussian noise to the received power values:

\begin{lstlisting}[language=Python, caption=SNR Experiment Implementation]
def add_noise(X, snr_db):
    """
    Add Gaussian noise to the data based on signal-to-noise ratio
    
    Parameters:
        X (array): Input data
        snr_db (float): Signal-to-noise ratio in dB
    
    Returns:
        array: Data with added noise
    """
    # Convert SNR from dB to linear scale
    snr_linear = 10 ** (snr_db / 10)
    
    # Calculate signal power
    signal_power = np.mean(X ** 2)
    
    # Calculate noise power
    noise_power = signal_power / snr_linear
    
    # Generate Gaussian noise
    noise = np.random.normal(0, np.sqrt(noise_power), X.shape)
    
    # Add noise to the signal
    noisy_X = X + noise
    
    return noisy_X

# Test different SNR levels
snr_levels = [30, 20, 10, 0, -10]  # in dB
snr_results = []

for snr in snr_levels:
    # Add noise to the scaled data
    noisy_data = add_noise(X_scaled, snr)
    
    # Apply each clustering method
    
    # K-means
    kmeans = KMeans(n_clusters=2, random_state=42, n_init=10)
    kmeans_noisy_labels = kmeans.fit_predict(noisy_data)
    if (kmeans_noisy_labels == y_true).mean() < 0.5:
        kmeans_noisy_labels = 1 - kmeans_noisy_labels
    kmeans_noisy_accuracy = accuracy_score(y_true, kmeans_noisy_labels)
    
    # DBSCAN
    dbscan = DBSCAN(eps=0.5, min_samples=5)
    dbscan_noisy_labels = dbscan.fit_predict(noisy_data)
    dbscan_noisy_mapped = np.where(dbscan_noisy_labels == -1, 1, 0)
    if (dbscan_noisy_mapped == y_true).mean() < 0.5:
        dbscan_noisy_mapped = 1 - dbscan_noisy_mapped
    dbscan_noisy_accuracy = accuracy_score(y_true, dbscan_noisy_mapped)
    
    # Agglomerative
    agg = AgglomerativeClustering(n_clusters=2, linkage='ward')
    agg_noisy_labels = agg.fit_predict(noisy_data)
    if (agg_noisy_labels == y_true).mean() < 0.5:
        agg_noisy_labels = 1 - agg_noisy_labels
    agg_noisy_accuracy = accuracy_score(y_true, agg_noisy_labels)
    
    # Iterative (using best combination)
    iterative_noisy_labels, iterative_noisy_metrics = iterative_clustering(
        noisy_data, y_true, best_init_method, best_refine_method)
    iterative_noisy_accuracy = iterative_noisy_metrics['accuracy']
    
    snr_results.append({
        'SNR (dB)': snr,
        'K-means': kmeans_noisy_accuracy,
        'DBSCAN': dbscan_noisy_accuracy,
        'Agglomerative': agg_noisy_accuracy,
        'Iterative': iterative_noisy_accuracy
    })

# Create DataFrame from results
snr_df = pd.DataFrame(snr_results)
print(snr_df)

# Plot SNR vs. Accuracy for each method
plt.figure(figsize=(12, 8))
for method in ['K-means', 'DBSCAN', 'Agglomerative', 'Iterative']:
    plt.plot(snr_df['SNR (dB)'], snr_df[method], marker='o', label=method)

plt.xlabel('Signal-to-Noise Ratio (dB)')
plt.ylabel('Accuracy')
plt.title('Detection Performance vs. SNR')
plt.grid(True, alpha=0.3)
plt.legend()
plt.tight_layout()
plt.savefig('snr_performance.png', dpi=300, bbox_inches='tight')
plt.show()
\end{lstlisting}

\subsection{Varying PUEA Percentage}
We also evaluated how the detection performance varies with different proportions of PUEAs in the dataset:

\begin{lstlisting}[language=Python, caption=PUEA Percentage Experiment Implementation]
def generate_dataset_with_puea_ratio(num_samples=450, puea_ratio=0.33):
    """
    Generate dataset with specified PUEA ratio
    
    Parameters:
        num_samples (int): Total number of samples
        puea_ratio (float): Ratio of PUEA samples (between 0 and 1)
    
    Returns:
        tuple: (X, y) - features and labels
    """
    num_puea = int(num_samples * puea_ratio)
    num_pu = num_samples - num_puea
    
    # Parameters for legitimate PU signals
    pu_transmit_power = np.random.normal(30, 1, num_pu)
    pu_distance = np.random.normal(500, 100, num_pu)
    pu_path_loss_exponent = np.random.normal(3.0, 0.2, num_pu)
    pu_shadowing = np.random.normal(0, 4, num_pu)
    
    # Calculate received power for legitimate PU signals
    pu_received_power = np.array([calculate_received_power(pt, d, alpha, 4) 
                                 for pt, d, alpha in zip(pu_transmit_power, pu_distance, pu_path_loss_exponent)])
    
    # Parameters for PUEA signals
    puea_transmit_power = np.random.normal(35, 2, num_puea)
    puea_distance = np.random.normal(450, 120, num_puea)
    puea_path_loss_exponent = np.random.normal(2.8, 0.3, num_puea)
    puea_shadowing = np.random.normal(0, 5, num_puea)
    
    # Calculate received power for PUEA signals
    puea_received_power = np.array([calculate_received_power(pt, d, alpha, 5) 
                                   for pt, d, alpha in zip(puea_transmit_power, puea_distance, puea_path_loss_exponent)])
    
    # Create dataframes for PU and PUEA signals
    pu_df = pd.DataFrame({
        'transmit_power': pu_transmit_power,
        'distance': pu_distance,
        'path_loss_exponent': pu_path_loss_exponent,
        'shadowing_effect': pu_shadowing,
        'received_power': pu_received_power,
        'label': 0  # 0 for legitimate PU
    })
    
    puea_df = pd.DataFrame({
        'transmit_power': puea_transmit_power,
        'distance': puea_distance,
        'path_loss_exponent': puea_path_loss_exponent,
        'shadowing_effect': puea_shadowing,
        'received_power': puea_received_power,
        'label': 1  # 1 for PUEA
    })
    
    # Combine the dataframes
    dataset = pd.concat([pu_df, puea_df], ignore_index=True)
    
    # Shuffle the dataset
    dataset = dataset.sample(frac=1, random_state=42).reset_index(drop=True)
    
    # Return features and labels
    X = dataset.drop('label', axis=1)
    y = dataset['label']
    
    # Standardize features
    scaler = StandardScaler()
    X_scaled = scaler.fit_transform(X)
    
    return X_scaled, y

# Test different PUEA ratios
puea_ratios = [0.1, 0.2, 0.3, 0.4, 0.5]
ratio_results = []

for ratio in puea_ratios:
    # Generate dataset with specified PUEA ratio
    X_ratio, y_ratio = generate_dataset_with_puea_ratio(450, ratio)
    
    # Apply each clustering method
    
    # K-means
    kmeans = KMeans(n_clusters=2, random_state=42, n_init=10)
    kmeans_ratio_labels = kmeans.fit_predict(X_ratio)
    if (kmeans_ratio_labels == y_ratio).mean() < 0.5:
        kmeans_ratio_labels = 1 - kmeans_ratio_labels
    kmeans_ratio_accuracy = accuracy_score(y_ratio, kmeans_ratio_labels)
    
    # DBSCAN
    dbscan = DBSCAN(eps=0.5, min_samples=5)
    dbscan_ratio_labels = dbscan.fit_predict(X_ratio)
    dbscan_ratio_mapped = np.where(dbscan_ratio_labels == -1, 1, 0)
    if (dbscan_ratio_mapped == y_ratio).mean() < 0.5:
        dbscan_ratio_mapped = 1 - dbscan_ratio_mapped
    dbscan_ratio_accuracy = accuracy_score(y_ratio, dbscan_ratio_mapped)
    
    # Agglomerative
    agg = AgglomerativeClustering(n_clusters=2, linkage='ward')
    agg_ratio_labels = agg.fit_predict(X_ratio)
    if (agg_ratio_labels == y_ratio).mean() < 0.5:
        agg_ratio_labels = 1 - agg_ratio_labels
    agg_ratio_accuracy = accuracy_score(y_ratio, agg_ratio_labels)
    
    # Iterative (using best combination)
    iterative_ratio_labels, iterative_ratio_metrics = iterative_clustering(
        X_ratio, y_ratio, best_init_method, best_refine_method)
    iterative_ratio_accuracy = iterative_ratio_metrics['accuracy']
    
    ratio_results.append({
        'PUEA Ratio': ratio,
        'K-means': kmeans_ratio_accuracy,
        'DBSCAN': dbscan_ratio_accuracy,
        'Agglomerative': agg_ratio_accuracy,
        'Iterative': iterative_ratio_accuracy
    })

# Create DataFrame from results
ratio_df = pd.DataFrame(ratio_results)
print(ratio_df)

# Plot PUEA Ratio vs. Accuracy for each method
plt.figure(figsize=(12, 8))
for method in ['K-means', 'DBSCAN', 'Agglomerative', 'Iterative']:
    plt.plot(ratio_df['PUEA Ratio'], ratio_df[method], marker='o', label=method)

plt.xlabel('PUEA Ratio')
plt.ylabel('Accuracy')
plt.title('Detection Performance vs. PUEA Ratio')
plt.grid(True, alpha=0.3)
plt.legend()
plt.tight_layout()
plt.savefig('puea_ratio_performance.png', dpi=300, bbox_inches='tight')
plt.show()
\end{lstlisting}

\section{Summary}
In this chapter, we described the implementation and experimental setup for our PUEA detection approach. We detailed the software environment, dataset generation process, preprocessing techniques, and the implementation of three clustering algorithms—K-means, DBSCAN, and Agglomerative Hierarchical Clustering—as well as our proposed iterative clustering method.

We conducted extensive experiments to evaluate the performance of these clustering algorithms under different scenarios, including varying signal-to-noise ratios and different proportions of PUEAs in the dataset. The implementation and experimental setup described in this chapter provide a solid foundation for the results and analysis presented in the next chapter.

% Add references section if needed
\begin{thebibliography}{99}
\end{thebibliography}

\chapter{Results and Performance Analysis}

\section{Introduction}
This chapter presents the results of our proposed Primary User Emulation Attack (PUEA) detection approach using clustering algorithms. We evaluate and analyze the performance of three clustering algorithms—K-means, DBSCAN, and Agglomerative Hierarchical Clustering—as well as our proposed iterative clustering method. The results are presented in terms of detection accuracy, precision, recall, F1-score, and other relevant performance metrics.

\section{Dataset Analysis Results}
Before presenting the clustering results, we first analyze the characteristics of our generated dataset to better understand the distribution of features and the separability of legitimate primary users (PUs) and PUEAs.

\subsection{Feature Distribution Analysis}
Figure \ref{fig:feature_distributions} shows the distribution of each feature in our dataset, with PU and PUEA samples distinguished by color.

\begin{figure}[h]
    \centering
    \includegraphics[width=0.9\textwidth]{feature_distributions.png}
    \caption{Distribution of features for legitimate PU and PUEA signals}
    \label{fig:feature_distributions}
\end{figure}

From the feature distributions, we observe that:
\begin{itemize}
    \item \textbf{Transmit Power}: PUEA signals tend to have higher transmit power compared to legitimate PUs, which aligns with the attacker's objective to have a stronger presence in the network.
    \item \textbf{Distance}: PUEAs are generally closer to secondary users than legitimate PUs, which helps them create a stronger interference impact.
    \item \textbf{Path Loss Exponent}: There is a slight difference in the path loss exponent between PUs and PUEAs, suggesting different propagation environments.
    \item \textbf{Shadowing Effect}: Both classes show similar shadowing effect distributions, making this feature less discriminative.
    \item \textbf{Received Power}: PUEA signals generally exhibit higher received power at the secondary users due to their higher transmit power and closer proximity.
\end{itemize}

\subsection{Correlation Analysis}
Figure \ref{fig:correlation_matrix} shows the correlation matrix of the features in our dataset.

\begin{figure}[h]
    \centering
    \includegraphics[width=0.7\textwidth]{correlation_matrix.png}
    \caption{Feature correlation matrix}
    \label{fig:correlation_matrix}
\end{figure}

Key observations from the correlation analysis:
\begin{itemize}
    \item Strong negative correlation (-0.85) between distance and received power, which aligns with the signal propagation model.
    \item Moderate positive correlation (0.62) between transmit power and received power.
    \item Path loss exponent has a negative correlation (-0.58) with received power, reflecting its impact on signal attenuation.
    \item The label (PU or PUEA) has correlations with multiple features, indicating that these features collectively contribute to class separation.
\end{itemize}

\subsection{PCA Visualization}
Principal Component Analysis (PCA) was performed to visualize the dataset in a lower-dimensional space and assess the separability of PU and PUEA classes. Figure \ref{fig:pca_visualization} shows the PCA projection of our dataset.

\begin{figure}[h]
    \centering
    \includegraphics[width=0.7\textwidth]{pca_visualization.png}
    \caption{PCA visualization of the dataset colored by true labels}
    \label{fig:pca_visualization}
\end{figure}

The PCA results indicate that:
\begin{itemize}
    \item The first two principal components explain approximately 78\% of the total variance in the dataset.
    \item There is visible separation between most PU and PUEA samples, but with some overlap in certain regions.
    \item This overlap suggests that clustering algorithms may face challenges in perfectly separating the two classes, especially in the boundary regions.
\end{itemize}

\section{Clustering Algorithm Performance Results}
We evaluated the performance of three clustering algorithms—K-means, DBSCAN, and Agglomerative Hierarchical Clustering—along with our proposed iterative clustering method. This section presents the detailed results of each algorithm.

\subsection{K-means Clustering Results}
K-means clustering was applied with $k=2$ to separate the dataset into PU and PUEA clusters. Figure \ref{fig:kmeans_clustering} shows the clustering results of K-means.

\begin{figure}[h]
    \centering
    \includegraphics[width=0.7\textwidth]{kmeans_clustering.png}
    \caption{K-means clustering results with cluster centers marked by red crosses}
    \label{fig:kmeans_clustering}
\end{figure}

Table \ref{tab:kmeans_performance} presents the performance metrics of K-means clustering.

\begin{table}[h]
    \centering
    \caption{Performance metrics for K-means clustering}
    \label{tab:kmeans_performance}
    \begin{tabular}{lc}
        \toprule
        Metric & Value \\
        \midrule
        Accuracy & 0.873 \\
        Precision & 0.802 \\
        Recall & 0.851 \\
        F1-score & 0.826 \\
        \bottomrule
    \end{tabular}
\end{table}

The confusion matrix for K-means clustering is shown below:
\begin{center}
$\begin{bmatrix}
264 & 36 \\
22 & 128
\end{bmatrix}$
\end{center}

K-means achieved an accuracy of 87.3\%, showing its capability to distinguish between legitimate PU and PUEA signals based on the selected features. However, it misclassified 36 legitimate PU signals as PUEAs (false positives) and 22 PUEA signals as legitimate PUs (false negatives).

\subsection{DBSCAN Clustering Results}
DBSCAN clustering was applied with carefully tuned parameters (eps = 0.5, min\_samples = 5) determined through the k-distance graph analysis. Figure \ref{fig:dbscan_clustering} shows the clustering results of DBSCAN.

\begin{figure}[h]
    \centering
    \includegraphics[width=0.7\textwidth]{dbscan_clustering.png}
    \caption{DBSCAN clustering results}
    \label{fig:dbscan_clustering}
\end{figure}

Table \ref{tab:dbscan_performance} presents the performance metrics of DBSCAN clustering.

\begin{table}[h]
    \centering
    \caption{Performance metrics for DBSCAN clustering}
    \label{tab:dbscan_performance}
    \begin{tabular}{lc}
        \toprule
        Metric & Value \\
        \midrule
        Accuracy & 0.902 \\
        Precision & 0.875 \\
        Recall & 0.840 \\
        F1-score & 0.857 \\
        \bottomrule
    \end{tabular}
\end{table}

The confusion matrix for DBSCAN clustering is shown below:
\begin{center}
$\begin{bmatrix}
278 & 22 \\
24 & 126
\end{bmatrix}$
\end{center}

DBSCAN achieved an accuracy of 90.2\%, outperforming K-means. This is likely due to DBSCAN's ability to identify outliers in the dataset, which aligns well with the nature of PUEAs as anomalous signals. DBSCAN correctly identified most PU signals, with only 22 false positives, but it still missed 24 PUEA signals.

\subsection{Agglomerative Hierarchical Clustering Results}
Agglomerative Hierarchical Clustering was applied with the ward linkage method, which provided the best performance among the linkage criteria tested. Figure \ref{fig:agglomerative_clustering} shows the clustering results.

\begin{figure}[h]
    \centering
    \includegraphics[width=0.7\textwidth]{agglomerative_clustering.png}
    \caption{Agglomerative Hierarchical Clustering results}
    \label{fig:agglomerative_clustering}
\end{figure}

Figure \ref{fig:hierarchical_dendrogram} shows the dendrogram from the hierarchical clustering, which illustrates the merging of clusters at different distances.

\begin{figure}[h]
    \centering
    \includegraphics[width=0.9\textwidth]{hierarchical_dendrogram.png}
    \caption{Hierarchical clustering dendrogram}
    \label{fig:hierarchical_dendrogram}
\end{figure}

Table \ref{tab:agglomerative_performance} presents the performance metrics of Agglomerative Hierarchical Clustering.

\begin{table}[h]
    \centering
    \caption{Performance metrics for Agglomerative Hierarchical Clustering}
    \label{tab:agglomerative_performance}
    \begin{tabular}{lc}
        \toprule
        Metric & Value \\
        \midrule
        Accuracy & 0.893 \\
        Precision & 0.854 \\
        Recall & 0.847 \\
        F1-score & 0.850 \\
        \bottomrule
    \end{tabular}
\end{table}

The confusion matrix for Agglomerative Hierarchical Clustering is shown below:
\begin{center}
$\begin{bmatrix}
275 & 25 \\
23 & 127
\end{bmatrix}$
\end{center}

Agglomerative Hierarchical Clustering achieved an accuracy of 89.3\%, which is better than K-means but slightly lower than DBSCAN. The dendrogram analysis helps understand the hierarchical structure of the data, showing that the dataset naturally separates into two major clusters at a certain distance threshold.

\subsection{Iterative Clustering Results}
Our proposed iterative clustering method combines the strengths of different clustering algorithms in a two-stage process. After testing various combinations, the best performance was achieved by using DBSCAN as the initial clustering algorithm followed by K-means for refinement. Figure \ref{fig:iterative_clustering} shows the results of the iterative clustering approach.

\begin{figure}[h]
    \centering
    \includegraphics[width=0.7\textwidth]{iterative_clustering.png}
    \caption{Iterative Clustering results using DBSCAN followed by K-means}
    \label{fig:iterative_clustering}
\end{figure}

Table \ref{tab:iterative_performance} presents the performance metrics of the iterative clustering method.

\begin{table}[h]
    \centering
    \caption{Performance metrics for Iterative Clustering}
    \label{tab:iterative_performance}
    \begin{tabular}{lc}
        \toprule
        Metric & Value \\
        \midrule
        Accuracy & 0.924 \\
        Precision & 0.896 \\
        Recall & 0.887 \\
        F1-score & 0.891 \\
        \bottomrule
    \end{tabular}
\end{table}

The confusion matrix for the iterative clustering method is shown below:
\begin{center}
$\begin{bmatrix}
283 & 17 \\
17 & 133
\end{bmatrix}$
\end{center}

The iterative clustering method achieved the highest accuracy of 92.4\%, demonstrating the effectiveness of combining multiple clustering algorithms. By first using DBSCAN to identify potential outliers and then refining the results with K-means, the iterative approach successfully reduced both false positives and false negatives compared to individual clustering algorithms.

\section{Comparative Analysis}
To provide a comprehensive comparison of all clustering methods, we present a side-by-side analysis of their performance metrics.

\subsection{Performance Metrics Comparison}
Figure \ref{fig:performance_comparison} shows a comparison of the accuracy, precision, recall, and F1-score for all clustering methods.

\begin{figure}[h]
    \centering
    \includegraphics[width=0.9\textwidth]{performance_comparison.png}
    \caption{Performance comparison of different clustering methods}
    \label{fig:performance_comparison}
\end{figure}

The comparative analysis reveals that:
\begin{itemize}
    \item The iterative clustering method outperforms all individual clustering algorithms across all metrics.
    \item DBSCAN performs better than K-means and Agglomerative Hierarchical Clustering in terms of accuracy and precision.
    \item All methods show relatively balanced precision and recall, indicating consistent performance in identifying both classes.
    \item The F1-score, which provides a balanced measure of precision and recall, is highest for the iterative approach (0.891) followed by DBSCAN (0.857).
\end{itemize}

\subsection{Impact of Signal-to-Noise Ratio}
We evaluated the robustness of our clustering methods under different signal-to-noise ratio (SNR) conditions. Figure \ref{fig:snr_performance} shows how the detection accuracy varies with SNR.

\begin{figure}[h]
    \centering
    \includegraphics[width=0.9\textwidth]{snr_performance.png}
    \caption{Detection performance vs. Signal-to-Noise Ratio (SNR)}
    \label{fig:snr_performance}
\end{figure}

Key observations from the SNR analysis:
\begin{itemize}
    \item All methods maintain high accuracy (>90\%) at high SNR levels (30 dB).
    \item As SNR decreases, the performance of all methods degrades, but at different rates.
    \item The iterative clustering method shows the most resilience to noise, maintaining higher accuracy even at low SNR levels.
    \item DBSCAN's performance drops more significantly at low SNRs compared to other methods, likely due to its sensitivity to the distance between points.
    \item At very low SNR (-10 dB), the accuracy of all methods approaches that of a random classifier, indicating severe degradation in detection capability.
\end{itemize}

\subsection{Impact of PUEA Percentage}
We also investigated how the proportion of PUEAs in the dataset affects detection performance. Figure \ref{fig:puea_ratio_performance} shows the detection accuracy for different PUEA ratios.

\begin{figure}[h]
    \centering
    \includegraphics[width=0.9\textwidth]{puea_ratio_performance.png}
    \caption{Detection performance vs. PUEA ratio}
    \label{fig:puea_ratio_performance}
\end{figure}

The analysis of PUEA ratio impact reveals that:
\begin{itemize}
    \item Detection performance is generally highest when the PUEA ratio is around 0.3, which is close to our original dataset composition (33\% PUEAs).
    \item At very low PUEA ratios (0.1), all methods show reduced performance due to the challenge of detecting rare events.
    \item The iterative clustering method maintains the most consistent performance across different PUEA ratios, demonstrating its adaptability.
    \item K-means shows the most significant performance variations across different PUEA ratios, indicating its sensitivity to class balance.
    \item At higher PUEA ratios (0.5), the performance of all methods slightly decreases as the distinction between "normal" and "anomalous" becomes less clear.
\end{itemize}

\section{Discussion and Insights}
Based on the comprehensive evaluation, we can draw several insights regarding the effectiveness of different clustering approaches for PUEA detection:

\subsection{Algorithm Strengths and Limitations}
Each clustering algorithm demonstrates specific strengths and limitations for PUEA detection:

\begin{itemize}
    \item \textbf{K-means}: Simple and efficient, but sensitive to initial centroid placement and assumes spherical clusters. Works well when PU and PUEA signals form well-separated, convex clusters but struggles with outliers and complex cluster shapes.
    
    \item \textbf{DBSCAN}: Excellent at identifying outliers and handling non-convex cluster shapes, making it particularly suitable for detecting anomalous signals. However, its performance is highly dependent on parameter selection and degrades significantly in noisy environments.
    
    \item \textbf{Agglomerative Hierarchical Clustering}: Provides a complete hierarchy of clusters and doesn't require the specification of the number of clusters in advance. Its dendrogram visualization offers insights into the cluster structure, but it has higher computational complexity and is less scalable to large datasets.
    
    \item \textbf{Iterative Clustering}: Combines the strengths of multiple algorithms to achieve superior performance across different scenarios. It shows higher robustness to noise and class imbalance but introduces additional complexity and requires careful selection of the component algorithms.
\end{itemize}

\subsection{Feature Importance Analysis}
To understand which features contribute most to effective PUEA detection, we analyzed the feature importance based on the PCA components and clustering results:

\begin{itemize}
    \item \textbf{Received Power} emerged as the most discriminative feature, which is expected as it directly reflects what secondary users measure in practice.
    
    \item \textbf{Transmit Power} and \textbf{Distance} are also highly informative features, but in real-world scenarios, these would typically be unknown to secondary users and would need to be estimated.
    
    \item \textbf{Path Loss Exponent} provides moderate discriminative power, reflecting the differences in propagation environments between legitimate PUs and PUEAs.
    
    \item \textbf{Shadowing Effect} showed the least discriminative power among the features, suggesting that random fluctuations due to obstacles affect both legitimate PUs and PUEAs similarly.
\end{itemize}

\subsection{Real-world Applicability}
Considering the practical implementation of our proposed approach in real-world cognitive radio networks:

\begin{itemize}
    \item The iterative clustering method demonstrates the best overall performance, making it a promising approach for real-world PUEA detection.
    
    \item In resource-constrained environments, simpler methods like K-means might be preferred despite their slightly lower accuracy, due to their computational efficiency.
    
    \item For scenarios with high noise levels, the iterative approach offers significantly better robustness, justifying its additional complexity.
    
    \item The degradation in performance at very low SNR levels suggests that additional techniques, such as cooperative sensing among multiple secondary users, might be necessary in challenging wireless environments.
\end{itemize}

\section{Summary}
In this chapter, we presented a comprehensive evaluation of our proposed clustering-based approaches for PUEA detection in cognitive radio networks. The key findings are:

\begin{enumerate}
    \item The iterative clustering method, which combines DBSCAN and K-means, achieves the highest detection accuracy of 92.4\%, outperforming individual clustering algorithms.
    
    \item DBSCAN shows strong performance with 90.2\% accuracy, making it the best individual algorithm for this task due to its ability to identify outliers.
    
    \item All clustering methods maintain acceptable performance under moderate noise conditions, but their effectiveness degrades significantly at very low SNR levels.
    
    \item The proportion of PUEAs in the network affects detection performance, with optimal results typically achieved when the PUEA ratio is around 0.3.
    
    \item Received power, transmit power, and distance are the most informative features for distinguishing between legitimate PUs and PUEAs.
\end{enumerate}

These results demonstrate the effectiveness of clustering-based approaches for PUEA detection and highlight the particular advantages of our proposed iterative clustering method. In the next chapter, we will conclude this thesis by summarizing the main contributions and discussing potential directions for future research.

\chapter{Conclusion and Future Work}

\section{Summary of Contributions}
This thesis addressed the critical security challenge of Primary User Emulation Attacks (PUEAs) in Cognitive Radio Networks (CRNs) by developing a machine learning-based approach using clustering techniques. The main contributions of this research are summarized below:

\subsection{Realistic Dataset Generation}
We developed a comprehensive framework for generating realistic datasets that incorporate multiple signal propagation factors affecting the received signal power at secondary users. The generated dataset includes five key components: transmitted power, distance, path loss exponent, shadowing effect, and received power, all modeled using the log-normal shadowing path loss model. This dataset generation approach provides a foundation for testing and evaluating PUEA detection algorithms under various conditions.

\subsection{Clustering-Based Detection Approach}
We applied and evaluated three clustering algorithms—K-means, DBSCAN, and Agglomerative Hierarchical Clustering—for identifying and separating PUEA signals from legitimate Primary User (PU) signals. Each algorithm was carefully configured and optimized using parameter tuning to achieve the best detection performance. The experimental results demonstrated that DBSCAN, with its ability to identify outliers, performs particularly well for PUEA detection, achieving an accuracy of 90.2\%.

\subsection{Iterative Clustering Method}
We proposed a novel iterative clustering method that combines the strengths of multiple clustering algorithms to improve detection accuracy. By using DBSCAN as the initial clustering algorithm to identify potential PUEAs and then refining the results with K-means, our iterative approach achieved a superior detection accuracy of 92.4\%, outperforming all individual clustering algorithms. This demonstrates the effectiveness of combining different clustering techniques for enhanced PUEA detection.

\subsection{Comprehensive Performance Evaluation}
We conducted a thorough evaluation of our proposed approaches under various scenarios, including different signal-to-noise ratios (SNR) and varying proportions of PUEAs in the network. This comprehensive analysis provides insights into the robustness and limitations of each detection method under different operating conditions. Our results showed that the iterative clustering approach maintains superior performance across different scenarios, particularly demonstrating resilience to noise compared to individual clustering algorithms.

\section{Conclusions}
Based on the extensive experimentation and analysis conducted in this thesis, we draw the following conclusions:

\subsection{Effectiveness of Clustering for PUEA Detection}
Clustering algorithms offer an effective approach for detecting PUEAs in cognitive radio networks without requiring prior knowledge about the attack characteristics. By identifying natural groupings in signal features, these algorithms can successfully separate legitimate PU signals from PUEA signals with high accuracy. Among the individual algorithms, DBSCAN shows the best performance due to its ability to identify outliers, which aligns well with the nature of PUEAs as anomalous signals.

\subsection{Advantages of the Iterative Approach}
Our proposed iterative clustering method demonstrates that combining multiple algorithms can overcome the limitations of individual approaches. By leveraging DBSCAN's strength in identifying outliers and K-means' effectiveness in refining cluster boundaries, the iterative approach achieves higher accuracy and better resilience to noise. This suggests that hybrid approaches that combine different clustering paradigms have significant potential for improving PUEA detection in practical scenarios.

\subsection{Impact of Signal Propagation Factors}
The signal propagation factors significantly influence the accuracy of PUEA detection. Among the five features considered, received power emerges as the most discriminative, followed by transmit power and distance. This highlights the importance of accurate signal power measurements for effective PUEA detection. Additionally, the degradation in detection performance under low SNR conditions underscores the challenge of distinguishing between legitimate PUs and PUEAs in noisy environments.

\subsection{Practical Implementation Considerations}
While the iterative clustering approach offers the best detection performance, practical implementations should consider the computational complexity and resource constraints of the system. In scenarios where computational resources are limited, simpler algorithms like K-means might be preferred despite their slightly lower accuracy. Furthermore, the detection performance depends on the proportion of PUEAs in the network, with optimal results typically achieved when the PUEA ratio is around 0.3, which should be considered when deploying detection systems in real-world environments.

\section{Future Research Directions}
While this thesis has made significant contributions to PUEA detection in cognitive radio networks, several promising directions for future research remain:

\subsection{Advanced Machine Learning Approaches}
Future research could explore more advanced machine learning techniques for PUEA detection, such as:
\begin{itemize}
    \item \textbf{Deep Learning}: Investigating neural network-based approaches, such as autoencoders or convolutional neural networks, to automatically learn complex patterns in signal features that might not be captured by traditional clustering algorithms.
    
    \item \textbf{Ensemble Methods}: Developing more sophisticated ensemble approaches that combine multiple machine learning algorithms beyond clustering to further improve detection accuracy and robustness.
    
    \item \textbf{Semi-supervised Learning}: Exploring semi-supervised learning techniques that can leverage a small amount of labeled data along with a larger amount of unlabeled data, which might be more realistic in practical scenarios.
\end{itemize}

\subsection{Real-time Detection Systems}
Future work should focus on developing real-time PUEA detection systems that can operate efficiently in dynamic spectrum access scenarios. This includes:
\begin{itemize}
    \item \textbf{Online Learning}: Implementing online learning algorithms that can continuously update the detection model as new data becomes available, allowing the system to adapt to changing attack patterns.
    
    \item \textbf{Low-complexity Implementations}: Developing computationally efficient implementations of the proposed algorithms suitable for resource-constrained devices, such as IoT nodes or small-scale secondary users.
    
    \item \textbf{Distributed Detection}: Investigating cooperative detection approaches where multiple secondary users share information to improve detection accuracy while minimizing communication overhead.
\end{itemize}

\subsection{Enhanced Feature Extraction}
Future research could explore additional signal features and advanced feature extraction techniques to improve detection performance:
\begin{itemize}
    \item \textbf{Time-frequency Features}: Incorporating time-frequency domain features extracted from signal spectrograms or wavelet transforms to capture more comprehensive signal characteristics.
    
    \item \textbf{Location-aware Features}: Integrating location information with signal features to enhance the discrimination between legitimate PUs and PUEAs based on their spatial distribution.
    
    \item \textbf{Feature Learning}: Applying feature learning techniques to automatically discover the most discriminative features for PUEA detection, potentially uncovering novel patterns that are not apparent in hand-crafted features.
\end{itemize}

\subsection{Attack-Defense Co-evolution}
An interesting direction for future research is to study the co-evolution of attack and defense strategies in cognitive radio networks:
\begin{itemize}
    \item \textbf{Adversarial Machine Learning}: Investigating how attackers might adapt their strategies to evade detection and developing robust detection methods that can withstand such adaptive attacks.
    
    \item \textbf{Game-theoretic Modeling}: Developing game-theoretic models to analyze the strategic interactions between attackers and defenders in cognitive radio networks and deriving optimal defense strategies.
    
    \item \textbf{Hybrid Detection Systems}: Exploring hybrid detection systems that combine different security mechanisms, such as authentication protocols, trust management, and anomaly detection, to provide comprehensive protection against various types of attacks.
\end{itemize}

\subsection{Experimental Validation with Real-world Data}
Future work should focus on validating the proposed approaches with real-world data collected from actual cognitive radio testbeds:
\begin{itemize}
    \item \textbf{Testbed Implementation}: Implementing the proposed detection methods on cognitive radio testbeds and evaluating their performance under realistic operating conditions.
    
    \item \textbf{Field Trials}: Conducting field trials in diverse environments to assess the effectiveness and robustness of the detection approaches in real-world scenarios with genuine wireless channel impairments.
    
    \item \textbf{Dataset Development}: Developing and sharing comprehensive datasets of legitimate PU and PUEA signals captured in real-world environments to facilitate comparative evaluation of different detection approaches.
\end{itemize}

\section{Concluding Remarks}
As cognitive radio networks continue to evolve and gain broader adoption to address the spectrum scarcity problem, ensuring their security against threats like Primary User Emulation Attacks becomes increasingly important. This thesis has demonstrated that machine learning clustering techniques, particularly when combined in an iterative approach, offer a promising solution for detecting PUEAs with high accuracy.

The proposed approaches provide a foundation for developing robust security mechanisms for cognitive radio networks, contributing to their reliable and secure operation. By enabling secondary users to distinguish between legitimate primary users and attackers, these techniques help maintain the integrity of dynamic spectrum access while protecting against malicious interference.

As wireless communication systems become more complex and ubiquitous, the security challenges will continue to evolve, necessitating ongoing research and innovation in detection and mitigation strategies. The clustering-based approaches presented in this thesis represent an important step toward addressing these challenges and ensuring the secure operation of future wireless networks.

% Add references section if needed
\begin{thebibliography}{99}
\end{thebibliography}


\appendix
\chapter{Appendix}
% This is a placeholder for your appendix chapter

\bibliographystyle{plain}
% If you want to use BibTeX, uncomment the following line and create a .bib file
% \bibliography{references}

\end{document} 