% Appendix A: Mathematical Derivations
\chapter{Mathematical Derivations}
\label{app:math}

\section{Signal Model Derivations}
\label{app:signal_model}

This appendix provides detailed mathematical derivations for the signal models used in the thesis.

\subsection{Primary User Signal Model}
The received signal at the secondary user when the primary user is present can be expressed as:
\begin{equation}
y(t) = h_{pu}(t) \cdot s_{pu}(t) + n(t)
\end{equation}
where:
\begin{itemize}
    \item $y(t)$ is the received signal
    \item $h_{pu}(t)$ is the channel gain from primary user to secondary user
    \item $s_{pu}(t)$ is the primary user transmitted signal
    \item $n(t)$ is the additive white Gaussian noise
\end{itemize}

\subsection{PUEA Signal Model}
When a malicious user performs a PUEA, the received signal becomes:
\begin{equation}
y(t) = h_{att}(t) \cdot s_{att}(t) + n(t)
\end{equation}
where $h_{att}(t)$ and $s_{att}(t)$ represent the channel gain and signal from the attacker, respectively.

\section{Clustering Algorithm Complexity Analysis}
\label{app:complexity}

\subsection{K-means Complexity}
The time complexity of K-means algorithm is:
\begin{equation}
O(n \cdot k \cdot d \cdot i)
\end{equation}
where:
\begin{itemize}
    \item $n$ is the number of data points
    \item $k$ is the number of clusters
    \item $d$ is the dimensionality of data
    \item $i$ is the number of iterations
\end{itemize}

\subsection{DBSCAN Complexity}
The time complexity of DBSCAN algorithm is:
\begin{equation}
O(n \log n)
\end{equation}
when using spatial indexing structures like R*-tree.

% Appendix B: Simulation Parameters
\chapter{Simulation Parameters}
\label{app:sim_params}

\section{Network Configuration}
\label{app:network_config}

\begin{table}[h]
\centering
\caption{Simulation Network Parameters}
\label{tab:network_params}
\begin{tabular}{@{}ll@{}}
\toprule
Parameter & Value \\
\midrule
Number of Primary Users & 5-20 \\
Number of Secondary Users & 10-50 \\
Number of Attackers & 1-5 \\
Coverage Area & 1000m × 1000m \\
Transmission Power (PU) & 1W \\
Transmission Power (SU) & 0.1W \\
Transmission Power (Attacker) & 0.5W \\
Path Loss Exponent & 2.5 \\
Shadowing Standard Deviation & 8 dB \\
Noise Figure & 7 dB \\
\bottomrule
\end{tabular}
\end{table}

\section{Feature Extraction Parameters}
\label{app:feature_params}

\begin{table}[h]
\centering
\caption{Feature Extraction Parameters}
\label{tab:feature_params}
\begin{tabular}{@{}ll@{}}
\toprule
Parameter & Value \\
\midrule
Sampling Frequency & 2 MHz \\
FFT Size & 1024 \\
Window Type & Hamming \\
Overlap & 50\% \\
Feature Vector Dimension & 12 \\
Observation Window & 100 ms \\
\bottomrule
\end{tabular}
\end{table}

\section{Clustering Algorithm Parameters}
\label{app:clustering_params}

\subsection{K-means Parameters}
\begin{table}[h]
\centering
\caption{K-means Algorithm Parameters}
\label{tab:kmeans_params}
\begin{tabular}{@{}ll@{}}
\toprule
Parameter & Value \\
\midrule
Number of Clusters (k) & 2-10 \\
Maximum Iterations & 300 \\
Tolerance & $10^{-4}$ \\
Initialization Method & k-means++ \\
Distance Metric & Euclidean \\
\bottomrule
\end{tabular}
\end{table}

\subsection{DBSCAN Parameters}
\begin{table}[h]
\centering
\caption{DBSCAN Algorithm Parameters}
\label{tab:dbscan_params}
\begin{tabular}{@{}ll@{}}
\toprule
Parameter & Value \\
\midrule
Epsilon ($\epsilon$) & 0.1-1.0 \\
Minimum Points (MinPts) & 3-10 \\
Distance Metric & Euclidean \\
\bottomrule
\end{tabular}
\end{table}

% Appendix C: Additional Results
\chapter{Additional Experimental Results}
\label{app:additional_results}

\section{Performance Comparison Tables}
\label{app:performance_tables}

This section provides detailed numerical results for all experimental scenarios.

\begin{table}[h]
\centering
\caption{Detection Accuracy Comparison (SNR = 5 dB)}
\label{tab:accuracy_5db}
\begin{tabular}{@{}lcccc@{}}
\toprule
Algorithm & Precision & Recall & F1-Score & Accuracy \\
\midrule
K-means & 0.842 & 0.798 & 0.819 & 0.834 \\
DBSCAN & 0.876 & 0.823 & 0.849 & 0.857 \\
Hierarchical & 0.821 & 0.789 & 0.805 & 0.818 \\
Enhanced K-means & 0.891 & 0.867 & 0.879 & 0.883 \\
Enhanced DBSCAN & 0.923 & 0.894 & 0.908 & 0.912 \\
\bottomrule
\end{tabular}
\end{table}

\begin{table}[h]
\centering
\caption{Detection Accuracy Comparison (SNR = 10 dB)}
\label{tab:accuracy_10db}
\begin{tabular}{@{}lcccc@{}}
\toprule
Algorithm & Precision & Recall & F1-Score & Accuracy \\
\midrule
K-means & 0.887 & 0.842 & 0.864 & 0.876 \\
DBSCAN & 0.912 & 0.878 & 0.895 & 0.901 \\
Hierarchical & 0.856 & 0.834 & 0.845 & 0.851 \\
Enhanced K-means & 0.934 & 0.908 & 0.921 & 0.926 \\
Enhanced DBSCAN & 0.956 & 0.932 & 0.944 & 0.948 \\
\bottomrule
\end{tabular}
\end{table}

\section{Statistical Significance Tests}
\label{app:statistical_tests}

Statistical significance of the experimental results was verified using paired t-tests with $\alpha = 0.05$.

\begin{table}[h]
\centering
\caption{P-values for Pairwise Algorithm Comparisons}
\label{tab:pvalues}
\begin{tabular}{@{}lcc@{}}
\toprule
Algorithm Pair & Detection Accuracy & False Alarm Rate \\
\midrule
Enhanced DBSCAN vs K-means & $< 0.001$ & $< 0.001$ \\
Enhanced DBSCAN vs DBSCAN & $< 0.01$ & $< 0.01$ \\
Enhanced K-means vs K-means & $< 0.001$ & $< 0.001$ \\
DBSCAN vs K-means & $< 0.05$ & $< 0.05$ \\
\bottomrule
\end{tabular}
\end{table}

% Appendix D: Source Code Structure
\chapter{Implementation Details}
\label{app:implementation}

\section{Algorithm Pseudocode}
\label{app:pseudocode}

\subsection{Enhanced DBSCAN Algorithm}
\begin{algorithm}[H]
\caption{Enhanced DBSCAN for PUEA Detection}
\label{alg:enhanced_dbscan}
\begin{algorithmic}[1]
\Require Feature matrix $X \in \mathbb{R}^{n \times d}$, parameters $\epsilon$, $MinPts$
\Ensure Cluster labels $C = \{c_1, c_2, \ldots, c_n\}$
\State Initialize all points as unvisited
\State $C \leftarrow 0$ (cluster counter)
\For{each point $p$ in $X$}
    \If{$p$ is unvisited}
        \State Mark $p$ as visited
        \State $NeighborPts \leftarrow regionQuery(p, \epsilon)$
        \If{$|NeighborPts| < MinPts$}
            \State Mark $p$ as noise
        \Else
            \State $C \leftarrow C + 1$
            \State $expandCluster(p, NeighborPts, C, \epsilon, MinPts)$
        \EndIf
    \EndIf
\EndFor
\State Apply post-processing for PUEA classification
\State \Return $C$
\end{algorithmic}
\end{algorithm}

\section{Performance Optimization Techniques}
\label{app:optimization}

The following optimization techniques were implemented to improve algorithm performance:

\begin{itemize}
    \item \textbf{Spatial Indexing}: Used KD-tree for efficient neighbor queries in DBSCAN
    \item \textbf{Feature Normalization}: Applied z-score normalization to all features
    \item \textbf{Dimensionality Reduction}: Used PCA when feature dimension > 10
    \item \textbf{Parallel Processing}: Implemented multi-threading for distance calculations
    \item \textbf{Memory Management}: Used efficient data structures to minimize memory usage
\end{itemize}

% Appendix E: Validation Data
\chapter{Validation Dataset Description}
\label{app:validation}

\section{Dataset Characteristics}
\label{app:dataset_chars}

The validation dataset consists of:
\begin{itemize}
    \item 10,000 legitimate primary user signals
    \item 5,000 PUEA signals with varying attack strategies
    \item 2,000 noise-only samples
    \item SNR range: -10 dB to 20 dB
    \item Sampling rate: 2 MHz
    \item Duration: 100 ms per sample
\end{itemize}

\section{Ground Truth Labels}
\label{app:ground_truth}

Each sample in the dataset is labeled as one of the following categories:
\begin{enumerate}
    \item \textbf{Legitimate PU (Class 0)}: Genuine primary user transmission
    \item \textbf{PUEA (Class 1)}: Primary user emulation attack
    \item \textbf{Noise (Class 2)}: No signal present (noise only)
\end{enumerate}

The distribution of classes in the validation dataset is designed to reflect realistic cognitive radio network scenarios.
