\chapter{Literature Review}

\section{Cognitive Radio Networks}
Cognitive Radio Networks (CRNs) represent an evolution in wireless communication systems that aims to address the growing problem of spectrum scarcity. Introduced by Mitola \cite{mitola1999cognitive} in 1999, cognitive radio technology enables intelligent and dynamic spectrum usage by allowing secondary users to access licensed spectrum bands when they are not being utilized by primary users. This opportunistic spectrum access paradigm is facilitated through key capabilities such as spectrum sensing, spectrum decision, spectrum sharing, and spectrum mobility \cite{akyildiz2006next}.

\subsection{Spectrum Sensing}
Spectrum sensing is the fundamental capability that allows cognitive radios to detect spectrum holes or white spaces. Various techniques have been developed for spectrum sensing, including energy detection, matched filter detection, cyclostationary feature detection, and cooperative sensing \cite{yucek2009survey}. Each of these techniques offers different trade-offs between detection accuracy, computational complexity, and implementation requirements.

\subsection{Dynamic Spectrum Access}
Dynamic Spectrum Access (DSA) enables secondary users to opportunistically access spectrum bands without causing harmful interference to primary users. DSA systems implement policies and protocols that govern when and how secondary users can access the spectrum, ensuring that primary users' communication rights are protected \cite{zhao2007survey}.

\section{Security Threats in Cognitive Radio Networks}
Despite their numerous benefits, CRNs are vulnerable to various security threats due to their open and dynamic nature. These threats can be classified into different categories based on the targeted layer of the network architecture \cite{fragkiadakis2013survey}.

\subsection{Physical Layer Threats}
Physical layer threats in CRNs include jamming attacks, primary user emulation attacks, and objective function attacks. These attacks directly target the fundamental operations of cognitive radios, particularly spectrum sensing and decision-making processes \cite{wang2010security}.

\subsection{MAC Layer Threats}
Medium Access Control (MAC) layer threats include control channel saturation, biasing sensing results in cooperative sensing, and selfish behavior in spectrum sharing. These attacks can disrupt the coordination among cognitive radio users and lead to inefficient spectrum utilization \cite{leon2010security}.

\section{Primary User Emulation Attacks}
Primary User Emulation Attack (PUEA) is one of the most severe threats to cognitive radio networks where a malicious entity mimics the signal characteristics of a legitimate primary user \cite{chen2008defense}.

\subsection{PUEA Models}
PUEA can be categorized into two main models: selfish PUEA and malicious PUEA \cite{chen2008defense}. In selfish PUEA, the attacker's objective is to maximize its own spectrum usage by forcing other secondary users to vacate the spectrum. In contrast, malicious PUEA aims to disrupt the entire cognitive radio network operation without necessarily benefiting from the spectrum usage.

\subsection{Impact of PUEA}
The impact of PUEA on cognitive radio networks is multifaceted. First, it reduces spectrum utilization efficiency by denying secondary users access to available spectrum bands. Second, it increases the false alarm rate in spectrum sensing, leading to unnecessary spectrum handoffs and service disruptions. Third, it can cause denial of service for legitimate secondary users, resulting in degraded network performance \cite{jin2010advanced}.

\section{PUEA Detection Techniques}
Various approaches have been proposed in the literature to detect and mitigate PUEAs. These approaches can be broadly classified into location-based, signal feature-based, and machine learning-based techniques.

\subsection{Location-based Detection}
Location-based detection techniques exploit the spatial information of transmitters to distinguish between legitimate primary users and attackers \cite{chen2008defense}. These techniques often rely on the deployment of location verification infrastructure, such as anchor nodes or direction-finding systems, to verify the claimed location of signal sources.

\subsection{Signal Feature-based Detection}
Signal feature-based detection techniques analyze various characteristics of received signals, such as energy level, modulation type, cyclic features, and channel impulse response, to identify anomalies that may indicate the presence of PUEAs \cite{yu2011optimal}.

\subsection{Machine Learning-based Detection}
Machine learning techniques have gained significant attention for PUEA detection due to their ability to learn from data and adapt to changing environments. Various supervised, unsupervised, and semi-supervised learning algorithms have been applied to this problem \cite{kawalec2019application}.

\subsubsection{Supervised Learning Approaches}
Supervised learning approaches for PUEA detection include support vector machines (SVM), artificial neural networks (ANN), and decision trees. These approaches require labeled training data containing both legitimate PU signals and PUEA signals \cite{tang2015spectrumwatch}.

\subsubsection{Unsupervised Learning Approaches}
Unsupervised learning approaches, such as clustering algorithms, do not require labeled training data and can identify natural groupings or patterns in signal data. K-means, DBSCAN, and hierarchical clustering have been applied to detect anomalies in spectrum sensing data that may indicate the presence of PUEAs \cite{rajasekar2015detection}.

\subsubsection{Semi-supervised Learning Approaches}
Semi-supervised learning approaches combine elements of both supervised and unsupervised learning, typically using a small amount of labeled data along with a larger amount of unlabeled data. These approaches are particularly suitable for PUEA detection in real-world scenarios where obtaining labeled data is challenging \cite{abdelaziz2018semi}.

\section{Clustering Algorithms for Anomaly Detection}
Clustering algorithms have been widely used for anomaly detection in various domains, including network security. In the context of PUEA detection, clustering algorithms can help identify outliers or unusual signal patterns that deviate from the normal behavior of legitimate primary users.

\subsection{K-means Clustering}
K-means clustering partitions observations into k clusters in which each observation belongs to the cluster with the nearest mean \cite{hartigan1979algorithm}. In PUEA detection, K-means can be used to separate legitimate PU signals from PUEA signals based on their feature vectors.

\subsection{DBSCAN Algorithm}
Density-Based Spatial Clustering of Applications with Noise (DBSCAN) is a density-based clustering algorithm that groups together points that are closely packed while marking points in low-density regions as outliers \cite{ester1996density}. DBSCAN is particularly suitable for PUEA detection as it can identify anomalies without requiring the number of clusters to be specified in advance.

\subsection{Agglomerative Hierarchical Clustering}
Agglomerative hierarchical clustering builds a hierarchy of clusters by progressively merging clusters \cite{murtagh2012algorithms}. This approach allows for the analysis of the cluster structure at different levels of granularity, which can be beneficial for understanding the relationship between different types of signals in PUEA detection.

\section{Research Gap and Contribution}
While various approaches have been proposed for PUEA detection, there is still a need for more effective and reliable detection methods, particularly in dynamic and uncertain wireless environments. Most existing works focus on specific aspects of PUEA detection and often make simplified assumptions about signal propagation models or attack scenarios.

This research contributes to the field by developing a comprehensive PUEA detection framework that:
\begin{enumerate}
    \item Generates realistic datasets incorporating multiple signal propagation factors.
    \item Applies and compares different clustering algorithms for PUEA detection.
    \item Proposes an iterative clustering method to improve detection accuracy.
    \item Evaluates the proposed approach under various attack scenarios and environmental conditions.
\end{enumerate}

% Add references section
\begin{thebibliography}{99}
\bibitem{mitola1999cognitive} Mitola, J., and Maguire, G. Q. (1999). Cognitive radio: making software radios more personal. IEEE Personal Communications, 6(4), 13-18.

\bibitem{akyildiz2006next} Akyildiz, I. F., Lee, W. Y., Vuran, M. C., and Mohanty, S. (2006). Next generation/dynamic spectrum access/cognitive radio wireless networks: A survey. Computer Networks, 50(13), 2127-2159.

\bibitem{yucek2009survey} Yucek, T., and Arslan, H. (2009). A survey of spectrum sensing algorithms for cognitive radio applications. IEEE Communications Surveys \& Tutorials, 11(1), 116-130.

\bibitem{zhao2007survey} Zhao, Q., and Sadler, B. M. (2007). A survey of dynamic spectrum access. IEEE Signal Processing Magazine, 24(3), 79-89.

\bibitem{fragkiadakis2013survey} Fragkiadakis, A. G., Tragos, E. Z., and Askoxylakis, I. G. (2013). A survey on security threats and detection techniques in cognitive radio networks. IEEE Communications Surveys \& Tutorials, 15(1), 428-445.

\bibitem{wang2010security} Wang, W., Li, H., Sun, Y., and Han, Z. (2010). Security in cognitive radio networks: An overview. In 2010 IEEE Wireless Communication and Networking Conference (pp. 1-6). IEEE.

\bibitem{leon2010security} Leon, O., Hernandez-Serrano, J., and Soriano, M. (2010). Securing cognitive radio networks. International Journal of Communication Systems, 23(5), 633-652.

\bibitem{chen2008defense} Chen, R., Park, J. M., and Reed, J. H. (2008). Defense against primary user emulation attacks in cognitive radio networks. IEEE Journal on Selected Areas in Communications, 26(1), 25-37.

\bibitem{jin2010advanced} Jin, Z., Anand, S., and Subbalakshmi, K. P. (2010). Detecting primary user emulation attacks in dynamic spectrum access networks. In 2010 IEEE International Conference on Communications (pp. 1-5). IEEE.

\bibitem{yu2011optimal} Yu, F. R., Tang, H., Huang, M., Li, Z., and Mason, P. C. (2011). Defense against spectrum sensing data falsification attacks in mobile ad hoc networks with cognitive radios. In MILCOM 2011 Military Communications Conference (pp. 1241-1246). IEEE.

\bibitem{kawalec2019application} Kawalec, A., and Owczarek, R. (2019). Application of machine learning in cognitive radio networks. In 2019 Signal Processing: Algorithms, Architectures, Arrangements, and Applications (SPA) (pp. 127-132). IEEE.

\bibitem{tang2015spectrumwatch} Tang, L., Chen, Y., Hines, E. L., and Alouini, M. S. (2015). Spectrumwatch: A framework for primary user emulation attack detection in cognitive radio networks. In 2015 IEEE International Conference on Communication Workshop (ICCW) (pp. 1573-1578). IEEE.

\bibitem{rajasekar2015detection} Rajasekar, S., Karuppiah, M., and Manoharan, R. (2015). Detection of primary user emulation attacks in cognitive radio networks using support vector machine. In 2015 International Conference on Computing and Communications Technologies (ICCCT) (pp. 165-169). IEEE.

\bibitem{abdelaziz2018semi} Abdelaziz, A., Elsabrouty, M., and Muta, O. (2018). Semi-supervised learning for primary user emulation attack detection using spectral correlation. In 2018 IEEE Global Conference on Signal and Information Processing (GlobalSIP) (pp. 1-5). IEEE.

\bibitem{hartigan1979algorithm} Hartigan, J. A., and Wong, M. A. (1979). Algorithm AS 136: A k-means clustering algorithm. Journal of the Royal Statistical Society. Series C (Applied Statistics), 28(1), 100-108.

\bibitem{ester1996density} Ester, M., Kriegel, H. P., Sander, J., and Xu, X. (1996). A density-based algorithm for discovering clusters in large spatial databases with noise. In Kdd (Vol. 96, No. 34, pp. 226-231).

\bibitem{murtagh2012algorithms} Murtagh, F., and Contreras, P. (2012). Algorithms for hierarchical clustering: an overview. Wiley Interdisciplinary Reviews: Data Mining and Knowledge Discovery, 2(1), 86-97.
\end{thebibliography}
