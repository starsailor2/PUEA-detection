\chapter{Conclusion and Future Work}

\section{Summary of Contributions}
This thesis addressed the critical security challenge of Primary User Emulation Attacks (PUEAs) in Cognitive Radio Networks (CRNs) by developing a machine learning-based approach using clustering techniques. The main contributions of this research are summarized below:

\subsection{Realistic Dataset Generation}
We developed a comprehensive framework for generating realistic datasets that incorporate multiple signal propagation factors affecting the received signal power at secondary users. The generated dataset includes five key components: transmitted power, distance, path loss exponent, shadowing effect, and received power, all modeled using the log-normal shadowing path loss model. This dataset generation approach provides a foundation for testing and evaluating PUEA detection algorithms under various conditions.

\subsection{Clustering-Based Detection Approach}
We applied and evaluated three clustering algorithms—K-means, DBSCAN, and Agglomerative Hierarchical Clustering—for identifying and separating PUEA signals from legitimate Primary User (PU) signals. Each algorithm was carefully configured and optimized using parameter tuning to achieve the best detection performance. The experimental results demonstrated that DBSCAN, with its ability to identify outliers, performs particularly well for PUEA detection, achieving an accuracy of 90.2\%.

\subsection{Iterative Clustering Method}
We proposed a novel iterative clustering method that combines the strengths of multiple clustering algorithms to improve detection accuracy. By using DBSCAN as the initial clustering algorithm to identify potential PUEAs and then refining the results with K-means, our iterative approach achieved a superior detection accuracy of 92.4\%, outperforming all individual clustering algorithms. This demonstrates the effectiveness of combining different clustering techniques for enhanced PUEA detection.

\subsection{Comprehensive Performance Evaluation}
We conducted a thorough evaluation of our proposed approaches under various scenarios, including different signal-to-noise ratios (SNR) and varying proportions of PUEAs in the network. This comprehensive analysis provides insights into the robustness and limitations of each detection method under different operating conditions. Our results showed that the iterative clustering approach maintains superior performance across different scenarios, particularly demonstrating resilience to noise compared to individual clustering algorithms.

\section{Conclusions}
Based on the extensive experimentation and analysis conducted in this thesis, we draw the following conclusions:

\subsection{Effectiveness of Clustering for PUEA Detection}
Clustering algorithms offer an effective approach for detecting PUEAs in cognitive radio networks without requiring prior knowledge about the attack characteristics. By identifying natural groupings in signal features, these algorithms can successfully separate legitimate PU signals from PUEA signals with high accuracy. Among the individual algorithms, DBSCAN shows the best performance due to its ability to identify outliers, which aligns well with the nature of PUEAs as anomalous signals.

\subsection{Advantages of the Iterative Approach}
Our proposed iterative clustering method demonstrates that combining multiple algorithms can overcome the limitations of individual approaches. By leveraging DBSCAN's strength in identifying outliers and K-means' effectiveness in refining cluster boundaries, the iterative approach achieves higher accuracy and better resilience to noise. This suggests that hybrid approaches that combine different clustering paradigms have significant potential for improving PUEA detection in practical scenarios.

\subsection{Impact of Signal Propagation Factors}
The signal propagation factors significantly influence the accuracy of PUEA detection. Among the five features considered, received power emerges as the most discriminative, followed by transmit power and distance. This highlights the importance of accurate signal power measurements for effective PUEA detection. Additionally, the degradation in detection performance under low SNR conditions underscores the challenge of distinguishing between legitimate PUs and PUEAs in noisy environments.

\subsection{Practical Implementation Considerations}
While the iterative clustering approach offers the best detection performance, practical implementations should consider the computational complexity and resource constraints of the system. In scenarios where computational resources are limited, simpler algorithms like K-means might be preferred despite their slightly lower accuracy. Furthermore, the detection performance depends on the proportion of PUEAs in the network, with optimal results typically achieved when the PUEA ratio is around 0.3, which should be considered when deploying detection systems in real-world environments.

\section{Future Research Directions}
While this thesis has made significant contributions to PUEA detection in cognitive radio networks, several promising directions for future research remain:

\subsection{Advanced Machine Learning Approaches}
Future research could explore more advanced machine learning techniques for PUEA detection, such as:
\begin{itemize}
    \item \textbf{Deep Learning}: Investigating neural network-based approaches, such as autoencoders or convolutional neural networks, to automatically learn complex patterns in signal features that might not be captured by traditional clustering algorithms.
    
    \item \textbf{Ensemble Methods}: Developing more sophisticated ensemble approaches that combine multiple machine learning algorithms beyond clustering to further improve detection accuracy and robustness.
    
    \item \textbf{Semi-supervised Learning}: Exploring semi-supervised learning techniques that can leverage a small amount of labeled data along with a larger amount of unlabeled data, which might be more realistic in practical scenarios.
\end{itemize}

\subsection{Real-time Detection Systems}
Future work should focus on developing real-time PUEA detection systems that can operate efficiently in dynamic spectrum access scenarios. This includes:
\begin{itemize}
    \item \textbf{Online Learning}: Implementing online learning algorithms that can continuously update the detection model as new data becomes available, allowing the system to adapt to changing attack patterns.
    
    \item \textbf{Low-complexity Implementations}: Developing computationally efficient implementations of the proposed algorithms suitable for resource-constrained devices, such as IoT nodes or small-scale secondary users.
    
    \item \textbf{Distributed Detection}: Investigating cooperative detection approaches where multiple secondary users share information to improve detection accuracy while minimizing communication overhead.
\end{itemize}

\subsection{Enhanced Feature Extraction}
Future research could explore additional signal features and advanced feature extraction techniques to improve detection performance:
\begin{itemize}
    \item \textbf{Time-frequency Features}: Incorporating time-frequency domain features extracted from signal spectrograms or wavelet transforms to capture more comprehensive signal characteristics.
    
    \item \textbf{Location-aware Features}: Integrating location information with signal features to enhance the discrimination between legitimate PUs and PUEAs based on their spatial distribution.
    
    \item \textbf{Feature Learning}: Applying feature learning techniques to automatically discover the most discriminative features for PUEA detection, potentially uncovering novel patterns that are not apparent in hand-crafted features.
\end{itemize}

\subsection{Attack-Defense Co-evolution}
An interesting direction for future research is to study the co-evolution of attack and defense strategies in cognitive radio networks:
\begin{itemize}
    \item \textbf{Adversarial Machine Learning}: Investigating how attackers might adapt their strategies to evade detection and developing robust detection methods that can withstand such adaptive attacks.
    
    \item \textbf{Game-theoretic Modeling}: Developing game-theoretic models to analyze the strategic interactions between attackers and defenders in cognitive radio networks and deriving optimal defense strategies.
    
    \item \textbf{Hybrid Detection Systems}: Exploring hybrid detection systems that combine different security mechanisms, such as authentication protocols, trust management, and anomaly detection, to provide comprehensive protection against various types of attacks.
\end{itemize}

\subsection{Experimental Validation with Real-world Data}
Future work should focus on validating the proposed approaches with real-world data collected from actual cognitive radio testbeds:
\begin{itemize}
    \item \textbf{Testbed Implementation}: Implementing the proposed detection methods on cognitive radio testbeds and evaluating their performance under realistic operating conditions.
    
    \item \textbf{Field Trials}: Conducting field trials in diverse environments to assess the effectiveness and robustness of the detection approaches in real-world scenarios with genuine wireless channel impairments.
    
    \item \textbf{Dataset Development}: Developing and sharing comprehensive datasets of legitimate PU and PUEA signals captured in real-world environments to facilitate comparative evaluation of different detection approaches.
\end{itemize}

\section{Concluding Remarks}
As cognitive radio networks continue to evolve and gain broader adoption to address the spectrum scarcity problem, ensuring their security against threats like Primary User Emulation Attacks becomes increasingly important. This thesis has demonstrated that machine learning clustering techniques, particularly when combined in an iterative approach, offer a promising solution for detecting PUEAs with high accuracy.

The proposed approaches provide a foundation for developing robust security mechanisms for cognitive radio networks, contributing to their reliable and secure operation. By enabling secondary users to distinguish between legitimate primary users and attackers, these techniques help maintain the integrity of dynamic spectrum access while protecting against malicious interference.

As wireless communication systems become more complex and ubiquitous, the security challenges will continue to evolve, necessitating ongoing research and innovation in detection and mitigation strategies. The clustering-based approaches presented in this thesis represent an important step toward addressing these challenges and ensuring the secure operation of future wireless networks.

% Add references section if needed
\begin{thebibliography}{99}
\end{thebibliography}
