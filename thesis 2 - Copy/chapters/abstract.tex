% Abstract
\chapter*{Abstract}
\addcontentsline{toc}{chapter}{Abstract}

Cognitive Radio Networks (CRNs) represent a significant advancement in wireless communication, enabling dynamic spectrum access and efficient resource utilization. However, these networks face critical security vulnerabilities, particularly Primary User Emulation Attacks (PUEA) where malicious entities mimic primary user signals to disrupt legitimate spectrum access. This thesis addresses the critical challenge of accurately detecting PUEAs in CRNs through innovative clustering-based methodologies.

The research presents a comprehensive comparative analysis between traditional clustering algorithms (DBSCAN, K-means, Agglomerative, and Spectral clustering) and an enhanced approach that applies K-Nearest Neighbors (KNN) and Means algorithms within established clusters. This novel two-stage detection methodology significantly improves detection accuracy by refining cluster-based outlier identification. The study evaluates these approaches across three distinct spatial scenarios with varying distances between primary users and attackers (84.85, 42.43, and 21.21 units), under different path loss exponents (2-6), shadowing values (4-12), and varying attacker presence percentages (10\%-50\%).

Experimental results demonstrate that the enhanced detection approach consistently outperforms traditional clustering methods, particularly in challenging scenarios with close proximity between legitimate users and attackers. Statistical significance tests confirm the substantial improvements in detection rates, false alarm rates, and overall accuracy. The research shows that applying KNN within clusters yields optimal results in most scenarios, with performance improvements of up to 18\% in detection accuracy and 22\% reduction in false alarms compared to the best-performing traditional clustering method.

This thesis contributes valuable insights into PUEA detection optimization and provides practical recommendations for implementing robust security measures in cognitive radio networks. The findings emphasize the importance of context-aware detection methods that can adapt to varying network conditions and threat levels, paving the way for more secure and reliable cognitive radio deployments.

\vspace{1cm}
\noindent\textbf{Keywords:} Cognitive Radio Networks, Security, Primary User Emulation Attack, Clustering Algorithms, K-Nearest Neighbors, Detection Performance, Wireless Security
