\documentclass{beamer}
\usepackage{graphicx}
\usepackage{amsmath}
\usepackage{hyperref}
\usepackage{url}
\usepackage{xcolor}

% Theme choice
\usetheme{Madrid}
\usecolortheme{whale}

% Title information
\title{PUEA Detection}
\subtitle{Primary User Emulation Attack Detection in Cognitive Radio Networks}
\author{Presenter Name}
\date{\today}

\begin{document}

% Title slide
\begin{frame}
    \titlepage
\end{frame}

% Outline slide
\begin{frame}{Outline}
    \tableofcontents
\end{frame}

% Introduction
\section{Introduction}
\begin{frame}{Introduction}
    \begin{itemize}
        \item Cognitive Radio Networks (CRNs) enable dynamic spectrum access
        \item Primary User Emulation Attacks (PUEA) are a significant security threat
        \item Malicious users mimic primary user signals to deny service to legitimate secondary users
        \item Our research focuses on effective detection mechanisms for PUEA
    \end{itemize}
\end{frame}

% Background
\section{Background}
\begin{frame}{Cognitive Radio Fundamentals}
    \begin{itemize}
        \item Cognitive Radio: Intelligent wireless communication system
        \item Adapts to available spectrum holes
        \item Components:
        \begin{itemize}
            \item Spectrum sensing
            \item Spectrum management
            \item Spectrum mobility
            \item Spectrum sharing
        \end{itemize}
    \end{itemize}
\end{frame}

\begin{frame}{Primary User Emulation Attack}
    \begin{itemize}
        \item PUEA: Attacker mimics primary user signal characteristics
        \item Goals:
        \begin{itemize}
            \item Deny spectrum access to legitimate secondary users
            \item Disrupt the cognitive radio network operation
            \item Potentially launch other sophisticated attacks
        \end{itemize}
        \item Challenging to detect due to similarity with legitimate signals
    \end{itemize}
\end{frame}

% Related Work
\section{Related Work}
\begin{frame}{Existing Detection Approaches}
    \begin{itemize}
        \item Energy Detection Based Methods
        \item Location Verification Techniques
        \item Signal Feature Analysis
        \item Machine Learning Approaches
        \item Cooperative Detection Schemes
    \end{itemize}
\end{frame}

% Proposed Methodology
\section{Proposed Methodology}
\begin{frame}{Our Approach}
    \begin{itemize}
        \item Hybrid detection framework combining:
        \begin{itemize}
            \item Signal characteristic analysis
            \item Machine learning classification
            \item Cooperative sensing elements
        \end{itemize}
        \item Key innovations:
        \begin{itemize}
            \item Feature extraction optimization
            \item Reduced false alarm rate
            \item Low computational complexity
        \end{itemize}
    \end{itemize}
\end{frame}

\begin{frame}{System Architecture}
    % Placeholder for system architecture diagram
    \centering
    \begin{figure}
        \fbox{[System Architecture Diagram]}
        \caption{Overview of the PUEA Detection System}
    \end{figure}
\end{frame}

\begin{frame}{Detection Algorithm}
    \begin{enumerate}
        \item Signal reception and pre-processing
        \item Feature extraction
        \item Classification using trained ML model
        \item Decision making and alert generation
        \item Feedback for continuous improvement
    \end{enumerate}
    
    % Placeholder for algorithm pseudocode
    \begin{block}{Key Algorithm Steps}
        \begin{verbatim}
        1. Collect signal samples
        2. Extract features F1, F2, ..., Fn
        3. Apply classifier C
        4. If P(attack) > threshold
           then Raise alert
        5. Update detection parameters
        \end{verbatim}
    \end{block}
\end{frame}

% Results
\section{Results and Analysis}
\begin{frame}{Experimental Setup}
    \begin{itemize}
        \item Testbed configuration:
        \begin{itemize}
            \item Software-defined radio platforms
            \item Spectrum analyzer
            \item Processing server
        \end{itemize}
        \item Performance metrics:
        \begin{itemize}
            \item Detection accuracy
            \item False alarm rate
            \item Detection latency
            \item Computational efficiency
        \end{itemize}
    \end{itemize}
\end{frame}

\begin{frame}{Performance Analysis}
    % Placeholder for results graph
    \centering
    \begin{figure}
        \fbox{[Performance Comparison Graph]}
        \caption{Detection Performance Comparison}
    \end{figure}
    
    \begin{itemize}
        \item Our approach achieves 95\% detection rate
        \item False alarm rate reduced by 60\% compared to existing methods
        \item Operates effectively in low SNR scenarios
    \end{itemize}
\end{frame}

% Conclusion
\section{Conclusion}
\begin{frame}{Conclusion and Future Work}
    \begin{itemize}
        \item Successfully developed a robust PUEA detection framework
        \item Key contributions:
        \begin{itemize}
            \item Novel feature extraction technique
            \item Improved detection accuracy in challenging environments
            \item Low-complexity implementation suitable for practical deployment
        \end{itemize}
        \item Future work:
        \begin{itemize}
            \item Extension to cooperative networks
            \item Integration with other security mechanisms
            \item Real-time implementation optimization
        \end{itemize}
    \end{itemize}
\end{frame}

% References
\section{References}
\begin{frame}[allowframebreaks]{References}
    \begin{thebibliography}{9}
        \bibitem{ref1}
        Author1, et al. (2022).
        \textit{Survey on PUEA Detection Methods in Cognitive Radio Networks}.
        IEEE Transactions on Cognitive Communications.
        
        \bibitem{ref2}
        Author2, et al. (2023).
        \textit{Machine Learning for PUEA Detection}.
        Wireless Communications and Mobile Computing.
        
        \bibitem{ref3}
        Author3, et al. (2024).
        \textit{Secure Cognitive Radio Networks: Challenges and Solutions}.
        IEEE Communications Surveys \& Tutorials.
    \end{thebibliography}
\end{frame}

% Questions slide
\begin{frame}
    \centering
    \huge{Thank You!}
    \vspace{1cm}
    
    \Large{Questions?}
\end{frame}

\end{document}
